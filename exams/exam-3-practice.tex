\documentclass[12pt,fleqn,answers]{exam}

\usepackage{pifont}
\usepackage{dingbat}
\usepackage{amsmath,color}
\usepackage{fleqn}
\usepackage{epsfig}
\usepackage{upgreek}
%\usepackage{mathptm}
\newcommand{\reals}{\mathbf{R}}
\newcommand{\integers}{\mathbf{Z}}
\newcommand{\erf}{\mathrm{erf}}
%\usepackage[euler-digits,euler-hat-accent,T1]{eulervm}
\usepackage{fourier}
\addpoints
\boxedpoints
\pointsinmargin
\pointname{pts}
\shadedsolutions
\definecolor{SolutionColor}{rgb}{1,1,0.7}
\newcommand{\dotprod}{\, {\scriptzcriptztyle \stackrel{\bullet}{{}}}\,}
\begin{document}
\large
\vspace{0.1in}
\noindent\makebox[3.0truein][l]{{\bf MATH 202 Spring 2024}}
{\bf Name:}\hrulefill\
\noindent \makebox[3.0truein][l]{\bf Exam 3--Practice}
{\bf Row and Seat:}\hrulefill\
%\normalsize

\noindent \emph{“Piglet noticed that even though he had a Very Small Heart, it could hold a rather large amount of Gratitude.”}
\hfill {\sc Winnie-the-Pooh}

\noindent \textbf{Warning:} For your exam, you \emph{must} show all of your work, and you must convence me that you could 
work any similar question.  Here I mostly just give answers, but not my work. If you all have questions about how to 
solve any of these questions, please ask.
\begin{questions}

\question Find the \emph{convergence set} for each series

\begin{parts}

\part $\sum_{k=1}^\infty \frac{x^k}{k}$
\begin{solution} The series converges for $x \in [-1,1)$.
\end{solution}
\part $\sum_{k=1}^\infty \frac{x^k}{k^2}$
\begin{solution} The convergence set is $[-1,1]$.
\end{solution}
\part $\sum_{k=1}^\infty k! \frac{x^k}{28}$
\begin{solution} The convergence set is $\{0\}$.
\end{solution}
\part $\sum_{k=1}^\infty  \frac{x^k}{k!}$
\begin{solution} The convergence iset is $\reals$.
\end{solution}

\end{parts}
\question Use power series to find the numerical value of each limit. You might like to use the facts
\begin{align*}
\ln(1+x) &= 
 x\operatorname{-}\frac{{{x}^{2}}}{2}\operatorname{+}\frac{{{x}^{3}}}{3}\operatorname{-}\frac{{{x}^{4}}}{4}\operatorname{+}\frac{{{x}^{5}}}{5}\operatorname{-}\frac{{{x}^{6}}}{6}\operatorname{+} \cdots, \\
\sqrt{1+x} &= 1\operatorname{+}\frac{x}{2}\operatorname{-}\frac{{{x}^{2}}}{8}\operatorname{+}\frac{{{x}^{3}}}{16}\operatorname{-}\frac{5 {{x}^{4}}}{128}\operatorname{+}\frac{7 {{x}^{5}}}{256}\operatorname{-}\frac{21 {{x}^{6}}}{1024}\operatorname{+}\cdots, \\
\arctan(x) &= x\operatorname{-}\frac{{{x}^{3}}}{3}\operatorname{+}\frac{{{x}^{5}}}{5}\operatorname{-}\frac{{{x}^{7}}}{7}\operatorname{+}\frac{{{x}^{9}}}{9}\operatorname{+} \dots.
\end{align*}

\begin{parts}
\part $\displaystyle \lim_{x \to 0}{\frac{\log{\left( {{x}^{3}}\operatorname{+}1\right) }}{\sin{\left( 2 {{x}^{3}}\right) }}}$

\begin{solution}
For $x \neq 0$, we have
\begin{align*}
    \frac{\log{\left( {{x}^{3}}\operatorname{+}1\right) }}
    {\sin{\left( 2 {{x}^{3}}\right) }} 
      &= \frac{{{x}^{3}}\operatorname{-}\frac{{{x}^{6}}}{2}\operatorname{+}\frac{{{x}^{9}}}{3} + \cdots }{2 x^3-(4 x^9)/3+ \dots}\\
      &= \frac{1-x^3/2+x^6/3-x^9/4+ \cdots}
      {2\operatorname{-}\frac{4 {{x}^{6}}}{3} + \cdots}
\end{align*}
So 
\begin{equation*}    
    \lim_{x \to 0}{\frac{\log{\left( {{x}^{3}}\operatorname{+}1\right) }}{\sin{\left( 2 {{x}^{3}}\right) }}} = \frac{1}{2}.
\end{equation*}
\end{solution}

\part $\displaystyle \lim_{x \to 0} \frac{\arctan(x) - \sin(x)}{x^3}$

\begin{solution}
\begin{equation*}
\lim_{x \to 0} \frac{\arctan(x) - \sin(x)}{x^3} = -\frac{1}{6}.
\end{equation*}
\end{solution}

\part $\displaystyle \lim_{x \to 0} \frac{\arctan(x) - \sin(x)}{x^4}$

\begin{solution}
\begin{equation*}
\lim_{x \to 0^{+}} \frac{\arctan(x) - \sin(x)}{x^4} = -\infty.
\end{equation*}
\end{solution}
\end{parts}
\question [1] Use the \emph{ratio} test to determine of the series $\sum_{k=0}^\infty \frac{  \left(\frac{k}{3} \right)^k }{ k!}$ converges or diverges.
\begin{solution}%[2.6in]
Algebra is supposed to be easier than calculus, so let's start by simplifying
\begin{equation*}
\frac{ (\frac{k+1}{3})^{k+1}}{(k+1)!} \times \frac{k!}{ \left(\frac{k}{3} \right)^k} = 
\frac{1}{3} (\frac{k+1}{k})^k = \frac{1}{3}  \left(1 + \frac{1}{k} \right)^k.
\end{equation*}
So $\displaystyle  \lim_{x \to \infty} \frac{ (\frac{k+1}{3})^{k+1}}{(k+1)!} \times \frac{k!}{ (\frac{k}{3})^k} = \frac{\mathrm{e}}{3}$.
And so $\displaystyle \sum_{k=0}^\infty \frac{  \left(\frac{k}{3} \right)^k }{ k!}$ converges. 
\end{solution}



\question [5] Find the numerical value of  \(\displaystyle \sum_{k=4}^\infty \left(\frac{1}{2} \right)^k \).
\begin{solution}%[2.5in]
\[
 \sum_{k=4}^\infty \left(\frac{1}{2} \right)^k = \sum_{k=0}^\infty \left(\frac{1}{2} \right)^{k + 4}
  = \frac{1}{2^4} \sum_{k=0}^\infty \left(\frac{1}{2} \right)^{k} = \frac{2}{2^4} = \frac{1}{8}.
\]
\end{solution}


\question The de Jonqui\'ere function \(\operatorname{Li}_2\) can be defined by 
\(\displaystyle
  \operatorname{Li}_2(x) = \sum_{k=1}^\infty \frac{x^k}{k^2}
\) and $\mathrm{dom}(\operatorname{Li}_2) = (-1,1)$

\begin{parts}

\part [5] Find the \emph{numerical value} of \(\operatorname{Li}_2(0)\).

\begin{solution}%[2.5in]
\[
 \operatorname{Li}_2(0) = \sum_{k=1}^\infty \frac{0^k}{k^2} = 0.
\]
\end{solution}

\part [5] Find the \emph{radius of convergence} for the power series \( \sum_{k=1}^\infty \frac{x^k}{k^2}\).

\begin{solution}%[2.5in] 
We need to use the absolute ratio test; for \(x \neq 0\), we have
\[
  \lim_{k \to \infty} \left| \frac{x^{k+1}}{(k+1)^2} \frac{k^2}{x^k} \right| = |x|.
\]
So the sum converges for \(|x| < 1\); that is, the radius of convergence is 1.


\end{solution}

\part [5] Find the \emph{numerical value} of \(\operatorname{Li}_2^\prime(0)\). \textbf{Hint:}
\(
 \operatorname{Li}_2(x) = x + \frac{x^2}{4} +  \frac{x^3}{9} + \cdots.
\)
\end{parts}

\begin{solution}%[2.5in]
\[
\operatorname{Li}_2(0) = 1.
\] 
\end{solution}

%\newpage
\question Determine convergence or divergence of each series.  Fully
justify your work. This doesn't mean you need to use our definition of convergent; instead use
the theorems we've estiblished.
\begin{parts}
\part [5] \(\displaystyle \sum_{k=1}^\infty \frac{1}{k} \)


\begin{solution}%[2.5in] 
This is the harmonic series. It
diverges. Alternatively, the integral test applies and tells us that
the series diverges; the ratio test gives no information.
\end{solution}

\part[5]  \(\displaystyle \sum_{k=0}^\infty (-1)^k  \)


\begin{solution}%[2.5in] 
The sequence \(k \mapsto (-1)^k\) does not converge to
zero; therefore the series \(\sum_{k=0}^\infty (-1)^k \) diverges.
The alternating series test, the integral test, and the ratio test
either do not apply or they give no information.
\end{solution}

\part[5]   \(\displaystyle \sum_{k=0}^\infty \frac{(-1)^k}{k+1} \)

\begin{solution}%[2.5in]  
This is a convergent alternating series.  We need to 
check three things.
\begin{itemize}

\item[\ding{52}] Is \(k \to \frac{1}{1+k}\) a positive sequence? {\bf Yes},
\item [\ding{52}] Is \(k \to \frac{1}{1+k}\) a decreasing sequence? 
{\bf Yes}

\item [\ding{52}] does  \(k \to \frac{1}{1+k}\) converge to zero? {\bf Yes}
\end{itemize}
\end{solution}



\part[5]  \(\displaystyle \sum_{k=0}^\infty \frac{1}{k!} \)

\begin{solution}%[2.5in]
This is a series of positive terms; we'll try the ratio
test.  We have
\[
  \lim_{k \to \infty} \frac{k!}{(k+1)!} = \lim_{k \to \infty}
  \frac{1}{1+k} = 0.
\]
Therefore, the series converges.
\end{solution}

\end{parts}


\question [5] My friend Milhous  claims that the sum \(\sum_k f_k\) converges provided
that the sequence \(f\) converges to zero. Show Milhous an example of a seqence \(f\)
that converges to zero, but the  sum \(\sum_k f_k\) diverges.

\begin{solution} The sequence \(k \in \integers_{>0} \mapsto \frac{1}{k}\) converges
to zero, but the harmonic sum \(\sum_{k = 1}^\infty \frac{1}{k}\) diverges.
\end{solution}

\question For all real numbers $x$, we have
$\displaystyle
   \sin(x) = \sum_{k=0}^\infty \frac{(-1)^k}{(2k+1)!} x^{2 k + 1}.
$

\begin{parts}

    \part [2] Find the power series representation for $\sin(x) - x$
    centered at zero. \textbf{Hint:} When you don't know where to start, go to your
    happy place: write the first few terms of the Taylor series for sine
    centered at zero. Then subtract $x$.

    \begin{solution}%[3.05in] 
    Let's go to our happy place; for all real $x$, we have
        \begin{equation*}
            \sin(x) = x - \frac{1}{6} x^3 + \frac{1}{120} x^5 + \cdots.
        \end{equation*}
        Thus 
        \begin{equation*}
            \sin(x) - x =  - \frac{1}{6} x^3 + \frac{1}{120} x^5  + \cdots.
        \end{equation*}
        Arguably this answer is OK, but the ellipsis (that is the $\cdots$) leaves
        too much to the imagination.  An explicit answer is
    \begin{equation*}
        \sin(x) - x =  \sum_{k=1}^\infty \frac{(-1)^k}{(2k+1)!} x^{2 k + 1}.
    \end{equation*}
\end{solution}
    \part[2]  For $x \neq 0$, find the \emph{first two nonzero terms} 
    in a power series representation for $\frac{\sin(x) - x}{x^3}$.
    Again, try visiting your happy place.
    \begin{solution}%[2.5in]
        Here is my happy place:  
        \begin{equation*}
        \sin(x) - x =  - \frac{1}{6} x^3 + \frac{1}{120} x^5 \cdots
        \end{equation*}
    For $x \neq 0$, we have 
    \begin{equation*}
        \frac{\sin(x) - x}{x^3} =  - \frac{1}{6} + \frac{1}{120} x^2 \cdots
        \end{equation*}
    And explicitly, we have
    \begin{equation*}
        \frac{\sin(x) - x}{x^3} =  \sum_{k=1}^\infty \frac{(-1)^k}{(2k+1)!} x^{2 k - 2}
        \end{equation*}
    It's optional to do, but changing $k \to k + 1$ gives
    \begin{equation*}
        \frac{\sin(x) - x}{x^3} =  \sum_{k=0}^\infty -\frac{(-1)^k}{(2k+3)!} x^{2 k}
        \end{equation*}
    \end{solution}

\newpage
    \part [2] Use the above result to find the \emph{numerical value}
     of the
    limit 
    \begin{equation*}
        \lim_{x \to 0} \frac{\sin(x) - x}{x^3}.
    \end{equation*}

    \begin{solution}
        We have
        \begin{align*}
            \lim_{x \to 0} \frac{\sin(x) - x}{x^3} &= \lim_{x \to 0 }
            \sum_{k=0}^\infty -\frac{(-1)^k}{(2k+3)!} x^{2 k},\\
            &= \sum_{k=0}^\infty -\frac{(-1)^k}{(2k+3)!} 0^{2 k}, \\
            &= -\frac{1}{6}.            
        \end{align*}
     
    \end{solution}
        

\end{parts}

\question [1] Define a function $\erf$ by the definite integral $\erf(x) = \frac{2}{\sqrt{\uppi}} \large \int_0^x \mathrm{e}^{-t^2} \, \mathrm{d} t$.
Find a power series representation for $\erf$.  Find the radius of convergence of this power series.
\begin{solution}%[3.5in]
We have
\begin{align*}
\erf(x) &= \frac{2}{\sqrt{\uppi}} \large \int_0^x \mathrm{e}^{-t^2} \, \mathrm{d} t, \\
            &= \frac{2}{\sqrt{\uppi}} \large \int_0^x \sum_{k=0}^\infty \frac{(-1)^k}{k!}  t^{2 k} \, \mathrm{d} t, \\
            &= \frac{2}{\sqrt{\uppi}}  \sum_{k=0}^\infty \large \int_0^x \frac{(-1)^k}{k!}  t^{2 k} \, \mathrm{d} t, \\
            &= \frac{2}{\sqrt{\uppi}}  \sum_{k=0}^\infty\frac{(-1)^k}{k!} \large \int_0^x   t^{2 k} \, \mathrm{d} t, \\
            &= \frac{2}{\sqrt{\uppi}}  \sum_{k=0}^\infty\frac{(-1)^k}{k!} \frac{x^{2 k+1}}{2 k + 1}.
\end{align*}
We could use the ratio test, but the theory of power series tells us that the termwise integration does not change
the radius of convergence. So the radius of convergence is infinity.
\end{solution}

\end{questions}
\end{document}