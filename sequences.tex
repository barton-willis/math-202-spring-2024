\documentclass[portrait,fleqn,12pt]{beamer}
\usetheme[progressbar=frametitle]{metropolis}
\usepackage{amsmath}
\usepackage{fleqn,amssymb,shadow}
%\usepackage{pifont,enumerate}

\usepackage{float}
\usepackage{epsfig}
\usepackage{enumerate}

\raggedright
\newcommand{\fma}{{\rm fma}}
\newcommand{\reals}{\mathbf{R}}
\newcommand{\epsmach}{\varepsilon_m}
\newcommand{\mymax}{\mbox{max} \,\,}
\newcommand{\mydeg}{\mbox{deg}}
\newcommand{\dom}{\mbox{dom}}
\newcommand{\fl}{\mbox{fl}}
\newcommand{\rel}{\mbox{rel}}
\newcommand{\cond}{\mbox{cond}}
\newcommand{\bern}{\mathcal{B}}
\newcommand{\order}{\mathcal{O}}
\newcommand{\atan}{\rm{arctan}}

\newcommand{\bfa}{\mathbf{a}}
\newcommand{\bfb}{\mathbf{b}}
\newcommand{\bfzero}{\mathbf{0}}
\newcommand{\bfx}{\mathbf{x}}


\newenvironment{alphalist}
   {\begin{enumerate}[(a)]
       \addtolength{\itemsep}{0.0\itemsep}}
     {\end{enumerate}}

\newcommand{\complex}{\mathbf{C}}
%\newcommand{\reals}{\mathbf{R}}
\newcommand{\integers}{\mathbf{Z}}
\newenvironment{define}[1]{
  \textbf{Definition} #1}{}

\newenvironment{idefine}[2]{
  \index{#1}
  \textbf{Definition} #2}{\(\blacktriangleleft\)}

\newenvironment{myexample}[1]{
  \textbf{Example} #1}


\newenvironment{mynote}[1]{
  \textbf{Note} #1}

\usepackage{fourier}
\begin{document}

\begin{frame}
\begin{flushleft} 
\textbf{Class notes for sequences} \\
MATH 202 \\
\today 
\end{flushleft}


\emph{“It's the sides of the mountain that sustain life, not the top.”} \\
$\phantom{xxxx}$ \hfill {\sc Robert Pirsig} 
\end{frame}


\begin{frame}{Definitions}


\begin{define} A function from either \(\integers_{\geq 0}\) or \(\integers_{> 0}\)
to \(\reals\) is a \emph{real-valued sequence}.
\end{define}
\vfill

\end{frame}

\begin{frame}

\begin{itemize}

\item Our textbook modifies sequence with \emph{infinite}.

\item The identifiers \(i,j,k, \dots, n\) are   traditional names for the \emph{independent} variable of a sequence.

\item The identifiers \(a,b\), and \(c\) are traditional names for a
sequence.


\item An integer subscript on a sequence means function evaluation.
Thus, if \(a\) is a sequence, we have
\[
   a_1 = a(1), \quad a_2= a(2),  \quad a_k = a(k).
\]

\item If the domain of a sequence \(a\) is \(\integers_{>0}\), we say
\(a_1\) is its {\em first term}, \(a_2\) is its {\em second term}, and \dots.
If the domain is  \(\integers_{\geq 0}\), its first term is \(a_0\).

\item Our textbook (sometimes) surrounds a sequence with curly braces.
This is fru-fru clutter. And I don't like fru-fru clutter.

\end{itemize}

\end{frame}
\begin{frame}

\begin{myexample} Each of the following functions are sequences:

\begin{alphalist}
\item \(a(k)  = (-1)^k \) and $\dom(a)=\integers_{\geq 0}$.

\item \(\displaystyle b(k) = \begin{cases} 100 & \mbox{ if } k <
    10 \\ \frac{1}{k} & \mbox{ if } k \geq 10 \end{cases} \)  and $\dom(b)=\integers_{\geq 0}$.

\item  \(\displaystyle c(k) = \frac{1}{k!} \) and $\dom(c)=\integers_{\geq 0}$. 

\item  \(\displaystyle d(k)  = \sum_{\ell=1}^k \frac{1}{\ell}\) and  $\dom(d)=\integers_{\geq 1}$. We have
\(d_1 = 1\), \(d_2 = 3/2\), and \(d_3 = 1+1/2 + 1/3 = 11/6\).

\end{alphalist}
\end{myexample}

\end{frame}
\begin{frame}{Abject silliness}

If you \emph{only} know the first few terms of a sequence, you know
\textbf{nothing} about its subsequent terms. To illustrate, the first five
terms of the sequences
\(
  a_n = 2 n + 1
\)
and
\(
  b_n = \begin{cases} 2 n +1 & n \leq 5 \\  \sqrt{5} & n \geq 6. \end{cases}
\)
are identical, but \(a \neq b\). 

Our textbook has some questions that give the first few terms of a sequence and
asks you to guess the formula. Such questions are \textbf{abject silliness}.

\end{frame}
\begin{frame}{Convergence}

\begin{define} A sequence \(a\) converges provided:
\begin{alphalist}
\item there is number \(L\) such that
\item for \emph{each} positive number \(\varepsilon\) 
\item there is an  integer \(N\) such that
\item \label{dee} for all \(k \in \integers_{\geq N}\), we have
\(L  - \varepsilon < a_k\) and \(a_k < L + \varepsilon\).
\end{alphalist}
 An alternative to `\ref{dee}' is \(|a_k - L | < \varepsilon\)
for all \(k \in \integers_{> N}\).

When this is the case, we say the sequence \(a\) converges to
\(L\). This is expressed as either
\[
  \lim_{\infty} a = L \mbox{ or } \lim_{n \to \infty} a_n = L.
\]
\end{define}

In logician speak, we have
\begin{equation*}
\left(\exists L \in \reals\right) 
\left(\forall \varepsilon \in \reals_{>0}\right)
\left(\exists N \in \integers\right)
\left(\forall k \in \integers_{>N}\right)
\left(|a_k - L | <  \varepsilon \right).
\end{equation*}
%\item We have \( \{ a_k \mid k > M \} = \{a_{M+1}, a_{M+2}, \dots \}\).
\end{frame}

\begin{frame}{G is for graphical}

\begin{itemize}

\item A sequence converges provided its graph has a horizontal asymptote.

\item A  definition of a horizontal asymptote 
would hardly differ from the definition of convergence.

\end{itemize}

\end{frame}

\begin{frame}{Undefinition of convergence}

\begin{define} A sequence \(a\) diverges provided:
\begin{alphalist}
\item for every number \(L\) 
\item there is a positive number \(\varepsilon\) such that
\item for every integer \(N\) 
\item there is an integer \(k \in \integers_{> N}\)
such that either \(a_k < L - \varepsilon\) or \(L + \varepsilon < a_k\).
\end{alphalist}


\end{define}

\end{frame}
\begin{frame}{Fact}

  \begin{Theorem} If a sequence converges, its limit is
unique. 
  \end{Theorem}


This proposition is about the only fact that our
book doesn't stuff into \S9.1!

Since limits of sequences are unique, we can write things like
\begin{equation*}
  \lim_{\infty} a = 42
\end{equation*}
without disrespecting equality. (Suppose we could prove that a sequence 
$a$ converges to both 42 and to 107. Then we would write
\begin{equation*}
  \lim_{\infty} a = 42  \text{ and } \lim_{\infty} a = 107.
\end{equation*}
And that is a proof that the truth value $42 = 107$ is true.





\end{frame}
\end{document}
