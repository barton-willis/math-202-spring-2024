\documentclass[portrait,fleqn,12pt]{beamer}
\usetheme[progressbar=frametitle]{metropolis}
\usepackage{amsmath}
\usepackage{fleqn,amssymb,shadow}
\usepackage{pifont,enumerate}

\usepackage{float}
\usepackage{epsfig}
%\usepackage{pifont}
%\usepackage{dingbat,bbding}
\usepackage{pdfcolmk}
\usepackage{enumerate}
%\usepackage{wasysym,marvosym}

\raggedright
\newcommand{\fma}{{\rm fma}}
\newcommand{\reals}{\mathbf{R}}
\newcommand{\fpq}{\phantom{}_0{\rm F}_1}
\newcommand{\epsmach}{\varepsilon_m}
\newcommand{\mymax}{\mbox{max} \,\,}
\newcommand{\mydeg}{\mbox{deg}}
\newcommand{\dom}{\mbox{dom}}
\newcommand{\fl}{\mbox{fl}}
\newcommand{\rel}{\mbox{rel}}
\newcommand{\cond}{\mbox{cond}}
\newcommand{\bern}{\mathcal{B}}
\newcommand{\order}{\mathcal{O}}
\newcommand{\atan}{\rm{arctan}}

\newcommand{\bfa}{\mathbf{a}}
\newcommand{\bfb}{\mathbf{b}}
\newcommand{\bfzero}{\mathbf{0}}
\newcommand{\bfx}{\mathbf{x}}


\newenvironment{alphalist}
   {\begin{enumerate}[(a)]
       \addtolength{\itemsep}{-1.25\itemsep}}
     {\end{enumerate}}

%\newenvironment{enumerate}{
 % \begin{enumerate}[\bullet]
 %   \addtolength{\itemsep}{-1.0\itemsep}}
 % {\end{enumerate}}

\usepackage{mathptm,calc}
\newcounter{mysec}\setcounter{mysec}{0}
\newcommand{\mysec}{%\
\setcounter{mysec}{\value{mysec}+1}
\themysec}

\newcommand{\startsec}[1]{\setcounter{myalgo}{0}
 \begin{flushleft}
    {\bf \mysec \,\, #1}
 \end{flushleft}}

\newcounter{myalgo}\setcounter{myalgo}{0}
\newcommand{\myalgo}{%\
\setcounter{myalgo}{\value{myalgo}+1} \themyalgo}
\newcommand{\algo}[1]{\vspace{0.1in}
      \begin{flushleft}
         {\bf Algorithm \themysec.\hspace{-0.15in}\myalgo} {\em #1}
      \end{flushleft}
      \vspace{0.1in}}

\newenvironment{myalgorithm}[1]
  {\setcounter{myalgo}{\value{myalgo}+1} 
   \textbf{Algorithm \themysec.\hspace{-0.15in}\myalgo} {\em #1} \\ \vspace{0.15in}}
  
\newcounter{myfig}\setcounter{myfig}{0}
\newcommand{\myfig}{%\
\setcounter{myfig}{\value{myfig}+1} \themyfig}
\newcommand{\fig}[1]{\vspace{0.1in}
      \begin{flushleft}
         {\bf Figure \themysec.\hspace{-0.15in}\myfig} {\em #1}
      \end{flushleft}
    \vspace{0.1in}}

\newcommand{\complex}{\mathbf{C}}
%\newcommand{\reals}{\mathbf{R}}
\newcommand{\integers}{\mathbf{Z}}
\newcommand{\myarg}{\mbox{arg}}

\newcommand{\realpart}{\mbox{Re}}
\newcommand{\imagpart}{\mbox{Im}}

\newenvironment{define}[1]{
  \textbf{Definition} #1}{}

\newenvironment{idefine}[2]{
  \index{#1}
  \textbf{Definition} #2}{\(\blacktriangleleft\)}

%\newenvironment{example}[1]{
  %\textbf{Example} #1}

%\newenvironment{facts}{\textbf{Facts}}

%\newenvironment{fact}{\textbf{Fact}}

\newenvironment{mynote}[1]{
  \textbf{Note} #1}

\newcounter{thmcount}\setcounter{thmcount}{0}
\newcommand{\mythmcount}{%\
\setcounter{thmcount}{\value{thmcount}+1} \thethmcount}

\newenvironment{myprop}[1]{
  \textbf{Proposition}\mysec.\hspace{-0.15in}\mythmcount \, #1}

\newenvironment{exercies}{\textbf{Exercies}}




\begin{document}

\begin{frame}
\small
\begin{flushleft} 
{ \bf
Classnotes \\
MATH 202 \\
\today }


\emph{“It's the sides of the mountain that sustain life, not the top.”} \hfill
{\sc Robert M. Pirsig} 
\end{flushleft}
\end{frame}


\begin{frame}{Definitions}


\begin{define} A function from either \(\integers_{\geq 0}\) or \(\integers_{> 0}\)
to \(\reals\) is a \emph{real-valued sequence}.
\end{define}
\vfill

\end{frame}

\begin{frame}

\begin{itemize}

\item Our textbook modifies sequence with \emph{infinite}.

\item The identifiers \(i,j,k, \dots, n\) are   traditional names for the \emph{independent} variable of a sequence.

\item The identifiers \(a,b\), and \(c\) are traditional names for a
sequence.


\item An integer subscript on a sequence means function evaluation.
Thus if \(a\) is a sequence, we have
\[
   a_1 = a(1), \quad a_2= a(2),  \quad a_k = a(k).
\]

\item If the domain of a sequence \(a\) is \(\integers_{>0}\), we say
\(a_1\) is its {\em first term}, \(a_2\) is its {\em second term}, and \dots.
If the domain is  \(\integers_{\geq 0}\), its first term is \(a_0\).

\item Our textbook (sometimes) surrounds a sequence with curly braces.
This is fru-fru clutter. And I don't like fru-fru clutter.



%\item In \S8.1 sequences have signature \(\integers_{\geq 1} \to
%\reals\). In later sections, they have signature \(\integers_{\geq 0}
%\to \reals\).

\end{itemize}

\end{frame}
\begin{frame}

\begin{example} Each of the following functions are sequences:

\begin{alphalist}
\item \(a(k)  = (-1)^k \) and $\dom(a)=\integers_{\geq 0}$.

\item \(\displaystyle b(k) = \begin{cases} 100 & \mbox{ if } k <
    10 \\ \frac{1}{k} & \mbox{ if } k \geq 10 \end{cases} \)  and $\dom(b)=\integers_{\geq 0}$.

\item  \(\displaystyle c(k) = \frac{1}{k!} \) and  and $\dom(c)=\integers_{\geq 0}$. 

\item  \(\displaystyle d(k)  = \sum_{\ell=0}^k \frac{1}{\ell+1}\) and  $\dom(d)=\integers_{\geq 0}$. We have
\(d_0 = 1\), \(d_1 = 3/2\), and \(d_2 = 1+1/2 + 1/3 = 11/6\).


%\item \(e = k \in \integers \mapsto \left \lfloor 10 \left( 10^{k-1} \pi - \lfloor 10^{k-1}
%\pi \rfloor \right) \right \rfloor\).  We have \(e_0 = 3\), \(e_1 =
%1\), \(e_2 = 4\), \(e_3 = 1\), and \(e_4 = 5\). Can you guess a simple
%rule for the value of \(e\)?
\end{alphalist}
\end{example}

\end{frame}
\begin{frame}{Abject silliness}

%\begin{enumerate}
% \item  \(\lceil  x  \rceil \) rounds the number \(x\) up to the next
%integer. It's called the \emph{ceiling} function.
% \item \(\lfloor x  \rfloor \) rounds the number \(x\) down to the next
%integer. It's called the \emph{floor} function.
%\item We have
%\[
%   \lceil -1.7 \rceil = -1, \,\,\, \lfloor -1.7 \rfloor -2, \,\,\, 
%   \lceil 1.7 \rceil = 2, \,\,\, \lfloor 1.7 \rfloor  = 1.
%\]
%\end{enumerate}

If you \emph{only} know the first few terms of a sequence, you know
\textbf{nothing} about its subsequent terms. To illustrate, the first five
terms of the sequences
\(
  a_n = 2 n + 1
\)
and
\(
  b_n = \begin{cases} 2 n +1 & n \leq 5 \\  \sqrt{5} & n \geq 6. \end{cases}
\)
are identical, but \(a \neq b\). 

Our textbook has some questions that give the first few terms of a sequence and
asks you to guess the formula. Such questions are \textbf{abject silliness}.

\end{frame}
\begin{frame}{Convergence}

\begin{define} A sequence \(a\) converges provided:
\begin{alphalist}
\item there is number \(L\) such that
\item for \emph{each} positive number \(\varepsilon\) 
\item there is an  integer \(N\) such that
\item \label{dee} for all \(k \in \integers_{\geq M}\), we have
\(L  - \varepsilon < a_k\) and \(a_k < L + \varepsilon\).
\end{alphalist}
When this is the case, we say the sequence \(a\) converges to
\(L\). This is expressed as either
\[
  \lim_{\infty} a = L \mbox{ or } \lim_{n \to \infty} a_n = L.
\]
\end{define}

In logician speak, we have
\begin{equation*}
\left(\exists L \in \reals\right) 
\left(\forall \varepsilon \in \reals_{>0}\right)
\left(\exists N \in \integers\right)
\left(L  - \varepsilon < a_k \text{ and } a_k < L + \varepsilon  \right).
\end{equation*}
%\item We have \( \{ a_k \mid k > M \} = \{a_{M+1}, a_{M+2}, \dots \}\).
\end{frame}
\begin{frame}
\begin{itemize}
\item An alternative to `\ref{dee}' is \(|a_k - L | < \varepsilon\)
for all \(k \in \integers_{\geq N}\).

\item Graphically, a sequence \emph{converges} if its graph has a
horizontal asymptote towards infinity.

\end{itemize}
\end{frame}
\begin{frame}{G is for graphical}

\begin{itemize}

\item A sequence converges provided its graph has a horizontal asymptote.

\item A  definition of a horizontal asymptote 
would hardly differ from the definition of convergence.

\end{itemize}

\end{frame}

\begin{frame}{Undefinition of convergence}

\begin{define} A sequence \(a\) diverges provided:
\begin{alphalist}
\item for every number \(L\) 
\item there is a positive number \(\varepsilon\) such that
\item for every integer \(N\) 
\item \label{dee} there is an integer \(k \in \integers_{\geq N}\)
such that either \(a_k < L - \varepsilon\) or \(L + \varepsilon < a_k\).
\end{alphalist}


\end{define}

\end{frame}
\begin{frame}{Facts}

\noindent \textbf{Theorem} If a sequence converges, its limit is
unique. 
\begin{enumerate}
\item This proposition is about the only fact that our
book doesn't stuff into \S9.1!
\end{enumerate}


\end{frame}
\end{document}

\noindent {\bf Example} Show that the sequence \(a_n  = (-1)^n \) does not
converge.

\noindent {\bf Proof} Suppose the sequence does converge.  Then there
is a number \(L\) and an integer \(M\) such that
\[
     \{ (-1)^n  \mid  n > M\} \subset (L - 1/2, L + 1/2). 
\] 
\end{frame}
\begin{frame}
But for all integers \(M\), we have
\(\{ (-1)^n  \mid  \, n > M\}\)  \(= \{-1,1\}. \)
So
\[
  \{-1,1\} \subset (L - 1/2, L + 1/2).
\]
This isn't possible, because \((L - 1/2, L + 1/2)\) is an interval
with length 1 and the distance between -1 and 1 is 2; therefore
the sequence \(a\) diverges.

\noindent {\bf Example}  The sequence \(\displaystyle b_n = \begin{cases} 100 & \mbox{ if } n <
    10 \\ \frac{1}{n} & \mbox{ if } n \geq 10 \end{cases}  \mbox{ and }
\mbox{dom}(b) = \integers_{\geq 0}  \) converges.

\noindent{\bf Proof} Let \(\varepsilon\) be a positive number.  Choose 
\(M = \mbox{max}\{10, \lceil \frac{1}{\varepsilon} \rceil \}\). For
\(n > M\), we have
\[
  0 < b_n = \frac{1}{n} < \frac{1}{M} \leq \frac{1}{ \lceil
    \frac{1}{\varepsilon} \rceil} \leq \frac{1}{\frac{1}{\varepsilon}}
  = \varepsilon.
\]
Thus
\(
   \{b_n \mid n > M \} \subset (-\varepsilon, \varepsilon).
\)

\end{frame}
\begin{frame}

\noindent \textbf{Question} Modify the proof for the sequence
\(\displaystyle b_n = \begin{cases} 100 & \mbox{ if } n <
    100 \\ \frac{1}{n} & \mbox{ if } n \geq 100 \end{cases}  \mbox{ and }
\mbox{dom}(b) = \integers_{\geq 0}  \).


{\bf Proposition} [page 451] Let \(F : [1,\infty) \to \mathbf{R}.\) If 
\(\displaystyle \lim_{\infty} F \in \mathbf{R}\), the sequence \(a_n = F(n)\) 
converges. Further
\[
   \lim_{n \to \infty} a_n = \lim_{x \to \infty} F(x).
\]

{\bf Example} Let \(a\) be the sequence \(a_n = (-1)^n / n\). It takes
some imagination, but we have
\[
   \frac{(-1)^n}{n} = \frac{\cos(\pi n)}{n}.
\]
Because \(\displaystyle \lim_{x \to \infty} \cos(\pi x) /x  = 0\), we have
\[
  \lim_{n \to \infty} \frac{ (-1)^n}{n} = 0.
\]

\end{frame}
\begin{frame}

{\bf Proposition} [Theorem A] Let \(a,b\) be convergent sequences. Then
\vspace{-0.4in}
\begin{alphalist}
\item \(a + b\) is a convergent sequence and 
\[
  \lim_{\infty} (a + b)  = \lim_{\infty} (a) + \lim_{\infty} (b).
\]
This says the limit is additive.

\item for any \(\alpha \in \mathbf{R}\), \(\alpha \,\, a\) is a convergent sequence and 
\[
  \lim_{\infty} (\alpha a)  = \alpha \lim_{\infty} (a).
\]
This says the limit is outative.

\item \(a  b\) is a convergent sequence and 
\[
  \lim_{\infty} (a  b)  = \lim_{\infty} (a)  \lim_{\infty} (b).
\]
This says the limit is distributes over products.

\end{alphalist}
\end{frame}
\begin{frame}

\begin{enumerate}

\item[(c)] If \(0 \notin \mbox{range}(b)\) and  
\(\displaystyle \lim_{\infty} (b) \neq 0\), then  \(a / b\) is a convergent sequence and 
\[
  \lim_{\infty} \left(\frac{a}{b}\right)  = \frac{\lim_{\infty} (a)}{\lim_{\infty} (b)}.
\]

Why the fuss over  \(0 \notin \mbox{range}(b)\)?  Well suppose  \(0
\in \mbox{range}(b)\). This means for some integer \(k\), we have
\(b_k  = 0\).  Yikes! \( \left(\frac{a}{b}\right)_k \) has no value
This means  \(\left(\frac{a}{b}\right)\) isn't a sequence.


\end{enumerate}
{\bf Definition}  Let \(a\) be a sequence. If for all \(n \in \mbox{dom}(a)\) we have
\begin{alphalist}
\item  \(a_{n+1} \leq a_n\), the sequence is \emph{monotone decreasing},
\item  \(a_{n+1} \geq a_n\), the sequence is \emph{ monotone increasing}.
\end{alphalist}
If a sequence is either monotone decreasing or increasing, it
is {\em monotone.\/}

\end{frame}
\begin{frame}

\begin{define} Let \(a\) be a sequence. If 
\begin{alphalist}
\item \( \mbox{range}(a)  \subset  (-\infty, U] \), we say \(U\)
  is  an  {\em upper\/} bound  for \(a\). We also say that \(a\) is
bounded above.
\item  \( \mbox{range}(a)  \subset  [L,\infty) \), we say \(L\)
  is  an  {\em lower\/} bound  for \(a\).  We also say that \(a\) is
bounded below.
\end{alphalist}
\end{define}

{\bf Proposition} [Theorem D] If a sequence is bounded above and monotone
increasing, it converges. Further if a sequence is bounded below and monotone
decreasing, it converges.  

\end{frame}
\begin{frame}

{\bf Example} The sequence \(c_n = \frac{1}{n!}\) converges. 

{\bf Proof} For all \(n \in \integers_{\geq 0}\) we have
\(
   c_n > 0 \mbox{ and } c_{n+1} \leq c_n.
\)
So \(c\) is monotone decreasing and bounded below by 0.  Thus \(c\) converges.

\begin{enumerate}
\item We \textbf{don't} know the limit of \(c\).  If I had to
guess, I'd say \(\displaystyle \lim_{ n \to \infty} 1 / n! =
0.\) But \textbf{nothing} in our proof tells us that this is the case.
\end{enumerate}

\startsec{Recursively defined sequences}

{\bf Definition} A sequence that is partially defined in terms of 
itself is said to be defined {\em recursively\/}.  

{\bf Example} Let  \(a\) be a sequence and suppose
\[
  a_k = \begin{cases} 1 & \mbox{ if } k = 0 \\
                      k a_{k-1} & \mbox{ if } k > 0
        \end{cases} .
\]
\end{frame}
\begin{frame}
Thus
\begin{eqnarray*}
a_0 &=& 1, \\
a_1 &=& 1 a_0 = 1, \\
a_2 &=& 2 a_1 = 2, \\
a_3 &=& 3 a_2 = 2 \times 3 = 6.
\end{eqnarray*}
It's not too hard to guess (and prove) that a non--recursive 
formula for \(a\) is 
\(
  a_k = k!.
\)
Not all recursively defined sequences have simple non-recursive
formulae.




\startsec{Geometric sums }

\textbf{Fact} For \(x \in \mathbf{R}\) and \(n \in \integers_{\geq 0}\),
we have
\[
  \sum_{k=0}^n x^k = 1 + x + x^2 + \cdots + x^n = \begin{cases} \frac{1-x^{n+1}}{1-x} & x \neq 1
    \\
    n + 1 & x = 1
\end{cases}.
\]

\end{frame}
\begin{frame}

\begin{enumerate}
 \item This is the famous (and useful) {\em geometric sum}.

 \item The sum index starts at {\em zero.\/}

 \item The first term in each sum is \(x^0\). For this formula  to 
be correct for \(x = 0\), we must demand that \(0^0 = 1\).

\end{enumerate}



\begin{frame}

\startsec{Sum index gymnastics}

For \(a,b,n \in \mathbf{Z}\) and for any sequence \(f\), we have
\[
  \sum_{k=a}^{k=b} f_k = \sum_{k+n = a}^{k+n = b} f_{k+n} =
  \sum_{k=a-n}^{b-n} f_{k+n}
\]
To generate the right side to this equality, {\em every\/} \(k\) was
replaced by \(k + n\). When \(b = \infty\), we have
\[
  \sum_{k=a}^{\infty} f_k = \sum_{k+n = a}^{k+n = \infty} f_{k+n} =
  \sum_{k=a-n}^{\infty} f_{k+n}
\]

{\bf Example} Find the value of \(\sum_{k=2}^\infty
\left(1/3\right)^k\).

We have
\[
   \sum_{k=2}^\infty \left(1/3 \right)^k =
  \sum_{k+2=2}^\infty \left(1/3\right)^{k+2}
  = \frac{1}{9} \sum_{k=0}^\infty \left(1/3 \right)^k
  = \frac{1}{9} \,\, \frac{3}{2} = \frac{1}{6}.
\]

\end{frame}
\end{frame}

\end{document}

 
\noindent{\bf Product notation}

{\bf Definition}  Let \(a\) be a sequence.  Define
\[
   \prod_{k=0}^n a_k = a_0 \, a_1 \, a_2 \dots a_n.
\]
When \(a\) is a constant sequence, say \(a_k = \alpha\) with
\(\alpha \in \mathbf{R} \), we write
\[
   \prod_{k=0}^n \alpha = \alpha^{n+1}.
\]
The lower {\em product index\/} needn't be 0; for example
\[
  \prod_{k=1}^n a_k = a_1 \, a_2 \, a_3 \dots a_n.
\]
What should we do with goofy looking things like
\(
  \prod_{k=1}^0 a_k ?
\)

{\bf Examples}

\begin{enumerate}

\item[(a)] \( \prod_{k=0}^{n-1} (k+1) = 1 \cdot 2 \cdot 3 \dots n = n!
  \)

\item[(b)] \( \prod_{k=0}^n k = 0. \)

\item[(c)] \(\prod_{k=0}^n 5 = 5^{n+1} \).

\end{enumerate}



\noindent{\bf \ding{95} Hypergeometric sequences}

{\bf Definition}  A sequence \(a\) is {\em hypergeometric\/} provided

\begin{enumerate}

\item[(a)] There is a {\em rational function\/} \(q : \subset
  \mathbf{R} \to \mathbf{R}\) such that 

\item[(b)] \(\mathbf{Z}^+ \subset \mbox{dom}(q) \) and

\item[(c)] the sequence \(a\) has a recursive definition 
\[
   a_{k+1} = \begin{cases} 1, & \mbox{ if } k = 0, \\
                          q(k) a_k, & \mbox{ if } k > 0
             \end{cases} .
\]
\end{enumerate}

{\bf Examples} 

\begin{enumerate} 

\item[(a)] Let \(q(x) = 1\).  This function is rational and it's domain
contains \(\mathbb{Z}^+\).  The sequence
\[
   a_{k+1} = \begin{cases} 1, & \mbox{ if } k = 0, \\
            a_k, & \mbox{ if } k > 0  \end{cases} 
\]
is hypergeometric.  A formula for \(a\)
\[
   a_k  = 1.
\]

\item[(b)] Let  \(q(x) = \frac{1}{x+1}\).  This function is rational and it's domain
contains \(\mathbb{Z}^+\).  The sequence
\[
   a_{k+1} = \begin{cases} 1, & \mbox{ if } k = 0, \\
            \frac{1}{k+1} a_k, & \mbox{ if } k > 0  \end{cases} 
\]
is hypergeometric.  Computing \(a_1, a_2\), and \(a_3\), it becomes 
apparent that a formula for \(a\)
\[
   a_k  = \frac{1}{k!}.
\]

\item[(c)] Let  \(q(x) = \frac{1}{(x+1)(x+\alpha)}\), where \(\alpha
  \in \mathbf{C}\).  This function is rational and provided \(-\alpha 
\notin \mathbb{Z}^+ \), the domain of \(q\)  contains
\(\mathbb{Z}^+\). The sequence
\[
   a_{k+1} = \begin{cases} 1, & \mbox{ if } k = 0, \\
            \frac{1}{(k+1)(k+\alpha)} a_k, & \mbox{ if } k > 0  \end{cases} 
\]
is hypergeometric.  Computing \(a_1, a_2\), and \(a_3\), it becomes 
apparent that a formula for \(a\)
\[
   a_k  = \frac{1}{k! (\alpha)_k}.
\]
In the last line, we introduced the {\em Pochhammer sequence}.  For any
\(\alpha \in \mathbf{C}\), define the Pochhammer sequence \((\alpha)\)
as
\[
   (\alpha)_n = \begin{cases} 1, \mbox{ if } n = 0, \\
                            \alpha (\alpha + 1) (\alpha + 2) \dots
                            (\alpha + n - 1), \mbox{ if }
                            n > 0.
               \end{cases}
\]

\end{enumerate}

{\bf Fact} If
\[
   a_{k+1} = \begin{cases} 1, & \mbox{ if } k = 0, \\
                          q(k) a_k, & \mbox{ if } k > 0
             \end{cases}
\]
a nonrecursive formula for \(a\) is 
\[
  a_n = \begin{cases}  1 &  \mbox{ if } n = 0, \\
         \prod_{k=1}^{n} q(k) & \mbox{ if } n > 0
        \end{cases}
\]
Using the convention 
\[
   \prod_{k=1}^{0} \mbox{anything} = 1,
\]
we can express the formula for \(a\) compactly as
\[
   a_n =  \prod_{k=1}^{n} q(k).
\] 

\clearpage

{\bf \ding{95} Hypergeometric functions}

A function \(F\) is said to be {\em hypergeometric\/} provided
in a neighborhood of zero, \(F\) has a convergent power-series
representation of the form
\[
   F(x) = \sum_{k=0}^\infty a_k x^k,
\]
where the sequence \(a\) is hypergeometric.

The ratio test provides a quick method for finding
the radius of convergence of any hypergeometric function.
Suppose
\[
   a_{k+1} = q_k a_k
\]
The series
\(
   \sum_{k=0}^\infty a_k x^k
\)
converges provided
\[
    | x \lim_{k \to \infty} q(k) | < 1
\]
If \(\lim_{k \to \infty} q(k) = 0\), the series converges on
\(\mathbf{R}\).



{\bf Examples}  

\begin{enumerate}

\item[(a)] Let \(a\) be the hypergeometric sequence
\( a_k = 1\). The function
\[
   F(x) = \sum_{k=0}^\infty x^k.
\]
is hypergeometric. Since
\[
  |x \lim_{k \to \infty}  1|  = |x|,
\]
the series converges on \((-1,1)\).

A simple formula for \(F\) is
\[
   F(x) = \frac{1}{1-x}.
\]
This function is hypergeometric.

\item[(b)] Let \(a\) be the hypergeometric sequence 
\( a_k = \frac{1}{k!} \).  The function
\[
   F(x) = \sum_{k=0}^\infty \frac{1}{k!} x^k.
\]
is hypergeometric. Since
\[
   |x \lim_{k \to \infty} \frac{1}{k!}| = 0,
\]
the series converges on \(\mathbf{R}\).

A simple formula for \(F\) is
\[
   F(x) = \exp(x).
\]
The natural exponential function is hypergeometric.


\item[(c)] Let \(Q\) be the rational function
\[
  Q(x) = \frac{1}{x (x + \alpha - 1)},
\]
where \(\alpha \in \mathbf{C} \) and \(\alpha \neq  \dots -3, -2, -1, 0 \). 

Let's find the sequence \(c\).  We have
\begin{eqnarray*}
  c_n &=& \prod_{k=1}^n  \frac{1}{k (k + \alpha -1)}  \\
       &=& \frac{1}{1 \cdot 2 \cdot 3 \cdots n}  \,\,\,
         \frac{1}{ \alpha \cdot (\alpha + 1) \cdot  \dots
           (\alpha + n -1)}, \\
       &=& \frac{1}{k! (\alpha)_n}
\end{eqnarray*}


%\begin{enumerate}

\item[\ding{43}] Why the restriction  \(\alpha \neq  \dots -3, -2, -1
  \)?  Yikes, for example \((-1)_1 = 0\), that's why.

\item[\ding{43}] We won't define the Pochhammer symbol noninteger or negative
subscripts (but we  could).

\end{enumerate}

The hypergeometric function associated with  \(Q(x) = \frac{1}{x (x +
  \alpha - 1)},\) is
\[
  F(x) = \sum_{k=0}^\infty \frac{1}{k! (\alpha)_k} x^k.
\]
The ratio test tells us that the series converges on all of \(\mathbf{R}\).


{\bf Question} Is there a name or a simple formula for \(F\)?  

{\bf Answer} Yes and sometimes. The standard identifier for \(F\) is
\[
   \phantom{F}_0 F_1(\alpha; x) = \sum_{k=0}^\infty \frac{1}{k! (\alpha)_k} x^k.
\]
For special  values of \(\alpha\),  it's possible to express
 \(\phantom{F}_0F_1\) in terms of trigonometric functions and square roots.
 
%\end{enumerate}
\end{document}