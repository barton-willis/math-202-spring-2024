\documentclass[portrait,fleqn,12pt]{beamer}
\usetheme[progressbar=frametitle]{metropolis}
\usepackage{amsmath}
\usepackage{fleqn,amssymb,shadow}
\usepackage{float}
\usepackage{epsfig}
\usepackage{enumerate}
\usepackage{cancel}
\usepackage{fontawesome5}
%\usepackage{fourier}
\usepackage{upgreek}
\raggedright

% Custom commands
\newcommand{\fma}{{\rm fma}}
\newcommand{\reals}{\mathbf{R}}
\newcommand{\epsmach}{\varepsilon_m}
\newcommand{\mymax}{\mbox{max} \,\,}
\newcommand{\mydeg}{\mbox{deg}}
\newcommand{\dom}{\mbox{dom}}
\newcommand{\fl}{\mbox{fl}}
\newcommand{\rel}{\mbox{rel}}
\newcommand{\cond}{\mbox{cond}}
\newcommand{\bern}{\mathcal{B}}
\newcommand{\order}{\mathcal{O}}
\newcommand{\atan}{\rm{arctan}}
\newcommand{\bfa}{\mathbf{a}}
\newcommand{\bfb}{\mathbf{b}}
\newcommand{\bfzero}{\mathbf{0}}
\newcommand{\bfx}{\mathbf{x}}
\newcommand{\complex}{\mathbf{C}}
\newcommand{\integers}{\mathbf{Z}}

% Custom environments
\newenvironment{alphalist}
   {\begin{enumerate}[(a)]
       \addtolength{\itemsep}{0.0\itemsep}}
     {\end{enumerate}}

\newenvironment{handlist}
   {\begin{enumerate}[\faHandPointRight]
       \addtolength{\itemsep}{0.0\itemsep}}
     {\end{enumerate}}

\newenvironment{define}[1]{
  \textbf{Definition} #1}{}

\newenvironment{idefine}[2]{
  \index{#1}
  \textbf{Definition} #2}{\(\blacktriangleleft\)}

\newenvironment{myexample}[1]{
  \textbf{Example} #1}


\newenvironment{mynote}[1]{
  \textbf{Note} #1}

\begin{document}

\begin{frame}
\begin{flushleft} 
\textbf{Can I do X?} \\
MATH 202 \\
\today 
\end{flushleft}


\emph{“The law is reason unaffected by desire.”} \hfill {\sc Aristotle } 
\end{frame}


\begin{frame}[fragile]{Everything which is not forbidden is allowed}

In the legal system, generally everything not forbidden is legal.  

\begin{handlist}
 \item In K-Town, city code prohibits backyard chickens, making  chickens illegal, 
 
 \item but painting your house purple is legal because it's not mentioned in city code.
 \end{handlist}
 
 However, I'm not a lawyer, so check before buying the purple house paint.
\end{frame}
 
\begin{frame}{Math is different}
 
\begin{handlist}
\item[\faHandPointRight] In math, most things not explicitly allowed are forbidden.
\item[\faHandPointRight] In math, a snarky answer to the question 
"Can I do X?"  is ``If there is an explicit  rule for it, sure; if not, likely no.''
\end{handlist}
 \vfill
\end{frame}

\begin{frame}{Undistracted by the question}

\begin{handlist}
\item To remain in power, politicians need to be \emph{undistracted by the question.}

\item Math teachers need this skill too.

\item A nonsnarky, but evasive,  answer to the question  "Can I do X?" is ``Maybe, but let's think about a strategy first.'' 

\item We'll return to strategy in a bit.
\end{handlist}
\end{frame}

\begin{frame}{Exhaustive rules}

In algebra, we attempt to enumerate everything that is allowed. If something isn't listed as a rule, likely it's not true.

\begin{handlist}
\item The list of rules of exponents in our QRS aims to be exhaustive.

\item  Often in math, we try  to condense rules to a minimal set, 

\item so sometimes something might be true, but not explicitly stated as a rule.
\end{handlist}

\end{frame}

\begin{frame}{Nonenumeratiuon of nonrules}

In the section on antideriatives, our book \textbf{doesn't} list
\begin{equation*}
 \left(\exists \text{ functions    } f,g \right) \left(
    \int f(x) g(x) \, \mathrm{d} x \neq  \int f(x) \, \mathrm{d} x 
     \times \int g(x) \, \mathrm{d} x \right)
\end{equation*}
as a nonrule. But it is a nonrule because 
\begin{equation*}
  \int x^2 x^3 \, \mathrm{d} x = \int x^5 \, \mathrm{d} x = \frac{1}{6} x^6
\end{equation*}
But
\begin{equation*}
  \int x^2 \mathrm{d} x \times \int x^3 \mathrm{d} x  = \frac{1}{12} x^7.
\end{equation*}

\begin{handlist}
\item Enumerating all nonrules would use up lots of paper.
\item So generally we don't try.
\end{handlist}
  
\end{frame}

\begin{frame}{What's a  rule?}

\begin{theorem}[multiplicative cancellation] We have 
\begin{equation*} 
   \left(\forall a, c\in \reals_{\neq 0}, b \in \reals \right)\left(\frac{a b}{a c} = \frac{b}{c} \right) \equiv \text{True}.
  \end{equation*}
\end{theorem}

\begin{handlist}
\item In words, this says that a common nonzero \emph{multiplicative} factor in a numerator and denominator can be ``canceled.''
\item Notationally, we write 
$
\frac{\cancel{a} b}{ \cancel{a} c}  = \frac{b}{c}.
$
\item Replacing $\frac{a b}{a c}$ by $\frac{b}{c}$ is generally regarded as a \emph{simplification.}

\end{handlist}

\end{frame}

\begin{frame}{Avoid slang}

\begin{handlist}

\item The verb ``cancel''  is mathematical slang.  
\item Slang is poorly defined, so it  often gets misused. An example of misuse is the \textbf{bogus cancellation}:
\begin{equation*}
   \frac{a+b}{a+c} =  \frac{\cancel{a}+b}{\cancel{a} +c} = \frac{1 + b}{1 + c}.
\end{equation*}
\item Our rule says that a common \emph{multiplicative factor} in the numerator and denominator can be canceled, but in this example, the common term is additive, not multiplicative.
\end{handlist}

\end{frame}

\begin{frame}{One bad apple}
\begin{theorem}
  \begin{equation*}
    \left(\exists a,b,c \in \reals_{>0}\right)
       \left(\frac{a+b}{a+c} \neq \frac{1 + b}{1 + c} \right) \equiv \mbox{True}.
 \end{equation*}
\end{theorem}
\begin{proof} Choose $a=2,b=3$, and $c=4$. We have
  \begin{equation*}
    \left[\frac{a+b}{a+c} \neq \frac{1 + b}{1 + c} \right] \equiv
    \left[\frac{2+3}{2+4} \neq \frac{1 + 3}{1 + 4} \right]  \equiv
    \left [\frac{5}{6} \neq \frac{4}{5} \right]  \equiv \mbox{True!}
  \end{equation*}
\end{proof}
\end{frame}

\begin{frame}{One bad apple redux}
  \begin{theorem}
    \begin{equation*}
      \left(\forall  a,b,c \in \reals_{>0}\right)
         \left(\frac{a+b}{a+c} = \frac{1 + b}{1 + c} \right) \equiv \mbox{False}.
   \end{equation*}
  \end{theorem}
  \begin{proof} The given statement is logically equivalent to 
    \begin{equation*}
      \left(\exists a,b,c \in \reals_{>0}\right)
      \left(\frac{a+b}{a+c} \neq \frac{1 + b}{1 + c} \right) \equiv \mbox{True}.
    \end{equation*}
    \end{proof}
  \end{frame}


\begin{frame}{Semantic matching}

To apply the multiplicative cancellation rule
\begin{equation*} 
   \left(\forall a, c\in \reals_{\neq 0}, b \in \reals \right)\left(\frac{a b}{a c} = \frac{b}{c} \right) \equiv \text{True}.
  \end{equation*}
 we don't need a literal match (a syntactic match) with $\frac{a b}{a c}$; rather $a,b$, and $c$ can match with any `blob,'  as long as
 the `blobs'  $a$ and $c$ are nonvanishing.
 
For example
\begin{equation*}
   \frac{ \cos(\uppi x) \cancel{ (z^2 + 1)}}{ \cancel{z^2 + 1} } =  \cos(\uppi x)
\end{equation*}
is legitimate.  We matched $a$ with $z^2 + 1$,  matched $b$ with $ \cos(\uppi x) $, and matched $c$ with 1.


\end{frame}

\begin{frame}{For every means for every}

Consider the statement:
\begin{equation*}
  \left(\forall a \in \reals_{\neq 0} \right)\left(\frac{a+b}{a} = 1 + b \right).
\end{equation*}

This statement is false. For instance, if we choose $a=2$ and $b=5$, then
\begin{equation*}
    \left[ \frac{2+5}{2} = 1+5 \right] \equiv   \left[ \frac{7}{2} = 6 \right]  \equiv \left[ 7 = 12\right]  \equiv \text{False}.
\end{equation*}

\begin{handlist}
\item Checking a special case for a ``for every''  statement is a powerful way to possibly show that it is false.
\end{handlist}

\end{frame}

\begin{frame}{Referential transparency}

\emph{Referential transparency} is a fancy term that means that we can substitute like for like without changing meaning.  

\begin{handlist}
\item Since for all real $x$, we have $x (x-1) = x^2 -x$, it's also the case that for all real $x$, we have
$$
  \cos(x (x-1)) = \cos(x^2 - x).
$$

\item Generally, mathematical notation is referentially transparent.  \textbf{But}

\item the question ``Does $\sum_{k=0}^\infty \frac{1}{2^k}$ converge?'' violates referential transparency. That's because
the question ``Does  $1 $ converge?'' is silly, yet  $\sum_{k=0}^\infty \frac{1}{2^k} = 1$ is true.
\end{handlist}

\end{frame}

\begin{frame}{Referential transparency minutia}


  Since `people' and `human' are synonyms, the statement
  \begin{quote}
     All people are created equal.
  \end{quote}
has the same meaning as
\begin{quote}
  All humans are created equal.
\end{quote}
But the statement
\begin{quote}
  Larry said, ``All people are created equal."
\end{quote}
could be true while 
\begin{quote}
  Larry said, ``All humans are created equal."
\end{quote}
could be false.


  
\end{frame}

\begin{frame}{Mathematical taxonomy}


Imagine that you are a turkey. If you can recognize that a tree is an oak (taxonomy), 
you have located a source of tasty acorns. It doesn't matter if the tree is a red or white oak--regardless
it's food.

Organizing mathematical concepts into categories (taxonomy), helps to suggest a 
solution strategy.
 Examples
\begin{handlist}


\item If  you can recognize an integrand is the product of a polynomial and an 
exponential function (taxonomy), use IBP.
\item To find a limit of an indeterminate form  (taxonomy), use the L'Hôpital rule.

\end{handlist}
\end{frame}


\end{document}

