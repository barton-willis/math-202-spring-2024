\documentclass[portrait,fleqn,12pt]{beamer}
\usetheme[progressbar=frametitle]{metropolis}
\usepackage{amsmath}
\usepackage{fleqn,amssymb,shadow}
\usepackage{float}
\usepackage{epsfig}
\usepackage{enumerate}
\usepackage{cancel}
\usepackage{fontawesome5}
\usepackage{fourier}

\raggedright

% Custom commands
\newcommand{\fma}{{\rm fma}}
\newcommand{\reals}{\mathbf{R}}
\newcommand{\epsmach}{\varepsilon_m}
\newcommand{\mymax}{\mbox{max} \,\,}
\newcommand{\mydeg}{\mbox{deg}}
\newcommand{\dom}{\mbox{dom}}
\newcommand{\fl}{\mbox{fl}}
\newcommand{\rel}{\mbox{rel}}
\newcommand{\cond}{\mbox{cond}}
\newcommand{\bern}{\mathcal{B}}
\newcommand{\order}{\mathcal{O}}
\newcommand{\atan}{\rm{arctan}}
\newcommand{\bfa}{\mathbf{a}}
\newcommand{\bfb}{\mathbf{b}}
\newcommand{\bfzero}{\mathbf{0}}
\newcommand{\bfx}{\mathbf{x}}
\newcommand{\complex}{\mathbf{C}}
\newcommand{\integers}{\mathbf{Z}}

% Custom environments
\newenvironment{alphalist}
   {\begin{enumerate}[(a)]
       \addtolength{\itemsep}{0.0\itemsep}}
     {\end{enumerate}}

\newenvironment{handlist}
   {\begin{enumerate}[\faHandPointRight]
       \addtolength{\itemsep}{0.0\itemsep}}
     {\end{enumerate}}

\newenvironment{define}[1]{
  \textbf{Definition} #1}{}

\newenvironment{idefine}[2]{
  \index{#1}
  \textbf{Definition} #2}{\(\blacktriangleleft\)}

\newenvironment{myexample}[1]{
  \textbf{Example} #1}


\newenvironment{mynote}[1]{
  \textbf{Note} #1}

\begin{document}

\begin{frame}
\begin{flushleft} 
\textbf{Can I do X?} \\
MATH 202 \\
\today 
\end{flushleft}


\emph{“The law is reason unaffected by desire.”} \hfill {\sc Aristotle } 
\end{frame}


\begin{frame}[fragile]{Everything which is not forbidden is allowed}

In the legal system, generally, everything not explicitly forbidden is legal.  

\begin{handlist}
 \item For example, in Kearney, backyard chickens are prohibited by city code, making them illegal, 
 
 \item but painting your house purple is legal because it's not mentioned in city code.
 \end{handlist}
 
 However, I'm not a lawyer, so \dots
\end{frame}
 
\begin{frame}{But math is different}
 
\begin{itemize}
\item[\faHandPointRight] In math, most things not explicitly allowed are forbidden.
\item[\faHandPointRight] In math, a pretty good answer to the answer to the question "Can I do X?"  is "if there is a rule for it, sure; if not, no way."
\end{itemize}
 \vfill
\end{frame}

\begin{frame}{Exhaustive rules}

In algebra, we attempt to enumerate everything that is allowed. If something isn't listed as a rule, likely it's not true.

\begin{handlist}
\item The enumeration of rules of exponents in our QRS aims to be exhaustive.

\item  But algebra tries to condense rules to a minimal set, so sometimes something might be provably true from a set of rules but not explicitly stated.
\end{handlist}

\end{frame}



\begin{frame}{Can I do \dots}

\begin{theorem}[multiplicative cancellation] We have 
\begin{equation*} 
   \left(\forall a, c\in \reals_{\neq 0}, c \in \reals \right)\left(\frac{a b}{a c} = \frac{b}{c} \right) \equiv \text{True}.
  \end{equation*}
\end{theorem}

\begin{handlist}
\item In words, this says that a common nonzero \emph{multiplicative} factor in a numerator and denominator can be "canceled."
\item Notationally, we write 
$
\frac{\cancel{a} b}{ \cancel{a} c}  = \frac{b}{c}.
$
\item Provided that $a$ and $c$ are nonzero, replacing $\frac{ab}{ac}$ by $\frac{b}{c}$ in any statement doesn't change its meaning (or truth value).
\end{handlist}

\end{frame}

\begin{frame}{Avoiding slang}

\begin{handlist}

\item The verb "canceled" is mathematical slang—its use is convenient but subject to abuse.  
\item A problem with slang is that it is often poorly defined and misused. An example of misuse is the \textbf{bogus cancellation}:
\begin{equation*}
   \frac{a+b}{a+c} =  \frac{\cancel{a}+b}{\cancel{a} +c} = \frac{b}{c}.
\end{equation*}
\item Our rule says that a common \emph{multiplicative factor} in the numerator and denominator can be canceled, but in this example, the common term is additive, not multiplicative.
\end{handlist}

\end{frame}

\begin{frame}{For every means for every}

Consider the statement:
\begin{equation*}
  \left(\forall a \in \reals_{\neq 0} \right)\left(\frac{a+b}{a} = 1 + b \right) \equiv \text{True}
\end{equation*}

This statement is false. For instance, if we choose $a=2$ and $b=5$, then
\begin{equation*}
    \left[ \frac{2+5}{2} = 1+5 \right] \equiv   \left[ \frac{7}{2} = 6 \right]  \equiv \left[ 7 = 12\right]  \equiv \text{False}.
\end{equation*}

\begin{handlist}
\item Checking a special case for a "for every" statement is a powerful way to possibly show that it is false.
\end{handlist}

\end{frame}
\end{document}

