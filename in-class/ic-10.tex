\documentclass[12pt,fleqn]{exam}
\usepackage{pifont,bbding,url}
\usepackage{upgreek}
\usepackage{amsmath}
\usepackage{fleqn}
\usepackage{epsfig}
\usepackage{pdfpages}
\usepackage[final]{microtype}
\usepackage{color}
\usepackage{enumerate}
%\usepackage[euler-digits,euler-hat-accent,T1]{eulervm}
\usepackage{amssymb, wasysym, enumerate}
\shadedsolutions
%\definecolor{SolutionColor}{rgb}{0.9,1,1}
\definecolor{SolutionColor}{rgb}{1,1,0.7}
\addpoints
\boxedpoints
\pointsinmargin
\pointname{pts}

\usepackage{fourier}
\newcommand{\dotprod}{\, {\scriptzcriptztyle \stackrel{\bullet}{{}}}\,}
\begin{document}

\newcommand{\reals}{\mathbf{R}}
\newcommand{\bi}{\mathbf{i}}
\newcommand{\bj}{\mathbf{j}}
\newcommand{\bk}{mathbf{k}}

\newcommand{\euler}{\mathrm{e}}
\newcommand{\ex}{1}
\newenvironment{alphalist}{
  \begin{enumerate}[(a)]
    \addtolength{\itemsep}{-1.0\itemsep}}
  {\end{enumerate}}

\newenvironment{handlist}{
  \begin{enumerate}[\leftthumbsup]
    \addtolength{\itemsep}{-1.0\itemsep}}
  {\end{enumerate}}


\vspace{0.1in}
\noindent\makebox[3.0truein][l]{{\bf MATH 202}}
{\bf Name:}\hrulefill\
\noindent \makebox[3.0truein][l]{\bf In class work 10}
{\bf Row:}\hrulefill\


\noindent \emph{“Study without desire spoils the memory, and it retains nothing that it takes in.”} \\ 
$\phantom{xxx}$ \hfill  \textsc{Leonardo da Vinci}

For all $a,b,x \in \reals$, we have 
\begin{align*}
\cos{\left( a x\right) } \cos{\left( b x\right) } & =\frac{\cos{\left( b x+a x\right) }}{2}+\frac{\cos{\left( b x-a x\right) }}{2},\\
\cos{\left( a x\right) } \sin{\left( b x\right) }&=\frac{\sin{\left( b x+a x\right) }}{2}+\frac{\sin{\left( b x-a x\right) }}{2}, \\
\sin{\left( a x\right) } \sin{\left( b x\right) } & =\frac{\cos{\left( b x-a x\right) }}{2}-\frac{\cos{\left( b x+a x\right) }}{2}
\end{align*}

\begin{questions}

\question  Find the numerical value of the definite integral

\begin{parts}

\part [1] $ \displaystyle  \int_0^{\pi/4} \sin(5 x) \sin(6 x) \,  dx$
\begin{solution}[3.0in]

\end{solution}

\part [1] $ \displaystyle  \int_0^{\pi} \sin(5 x) \sin(6 x) \,  dx$

\end{parts}
\newpage

\question Use the reduction formula
\begin{equation}
 \int \tan(x)^n \, \mathrm{d} x = \frac{\tan(x)^{n-1}}{n-1}  - \int \tan(x)^{n-2} \, \mathrm{d} x, \quad n \neq 1
\end{equation}
to find $\displaystyle \int \tan(28 x)^6 \, \mathrm{d} x$
\begin{solution}%[3.0in]

\end{solution}

\newpage

\question Use the reduction formula
\begin{equation}
 \int \sec(x)^n \, \mathrm{d} x = \frac{\sec(x)^{n-2} \tan(x)}{n-1}  + \frac{n-2}{n-1} \int \sec(x)^{n-2} \, \mathrm{d} x, \quad n \neq 1
\end{equation}
along with
\begin{equation}
  \int \sec(x) \, \mathrm{d} x =  \log \left( \tan{(x)}+\sec{(x)}\right) 
\end{equation}
to find $\displaystyle  \int \sec(28 x)^4 \, \mathrm{d} x$


\end{questions}
%\includepdf[pages={1-}]{cheat_sheet.pdf}     
\end{document}