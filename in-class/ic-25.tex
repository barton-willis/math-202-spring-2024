\documentclass[12pt,fleqn]{exam}
\usepackage{amssymb}
\usepackage[intlimits]{amsmath}
\usepackage{epsfig}
\usepackage{upgreek}
\usepackage[super]{nth}
\usepackage[colorlinks=true,linkcolor=black,anchorcolor=black,citecolor=black,filecolor=black,menucolor=black,runcolor=black,urlcolor=black]{hyperref}
\usepackage[letterpaper, margin=0.75in]{geometry}
\addpoints
\boxedpoints
\pointsinmargin
\pointname{pts}
\usepackage{tikz}
\usepackage{tkz-euclide}
\usetikzlibrary{shapes.geometric}
\usetikzlibrary{calc}
\usepackage[final]{microtype}
\frenchspacing
\usepackage[american]{babel}
\usepackage[T1]{fontenc}
\usepackage[]{fourier}

\usepackage{isomath}
\usepackage{upgreek,amsmath}
\usepackage{graphicx}

\newcommand{\dotprod}{\, {\scriptzcriptztyle\stackrel{\bullet}{{}}}\,}

\newcommand{\reals}{\mathbf{R}}
\newcommand{\lub}{\mathrm{lub}} 
\newcommand{\glb}{\mathrm{glb}} 
\newcommand{\complex}{\mathbf{C}}
\newcommand{\dom}{\mbox{dom}}
\newcommand{\range}{\mbox{range}}
\newcommand{\cover}{{\mathcal C}}
\newcommand{\integers}{\mathbf{Z}}
\newcommand{\vi}{\, \mathbf{i}}
\newcommand{\vj}{\, \mathbf{j}}
\newcommand{\vk}{\, \mathbf{k}}
\newcommand{\bi}{\, \mathbf{i}}
\newcommand{\bj}{\, \mathbf{j}}
\newcommand{\bk}{\, \mathbf{k}}
\DeclareMathOperator{\Arg}{\mathrm{Arg}}
\DeclareMathOperator{\Ln}{\mathrm{Ln}}
\DeclareMathOperator{\sign}{\mathrm{sign}}
\newcommand{\imag}{\, \mathrm{i}}
\newcommand{\erf}{\mathrm{erf}}
\newcommand{\e}{\mathrm{e}}

\usepackage{graphicx}
\usepackage{color}
%\shadedsolutions
%\definecolor{SolutionColor}{rgb}{1,0.72,0.46} %{0.8,0.9,1}
\newcommand\AM{\textsc{am}}
\newcommand\PM{\textsc{pm}}


     
\newcommand{\quiz}{25}
\newcommand{\term}{Spring}
\newcommand{\due}{Tuesday 30 April 13:20}
\newcommand{\class}{MATH 202, Spring \the\year}
\begin{document}
\large
\noindent\makebox[3.0truein][l]{\textbf{\class}}
\textbf{Name:} \hrulefill \\
\noindent \makebox[3.0truein][l]{\textbf{In class work  \quiz}}
\textbf{Row and Seat}:\hrulefill\\


\noindent \emph{“Finding Nirvana is like locating silence.”} \hfill {\sc Jack Kerouac}

\vspace{0.1in}
\noindent  In class work  \textbf{\quiz}  has questions \textbf{1} 
through  \textbf{\numquestions} \/ with a total of 
\textbf{\numpoints\/} points. Turn in your work at the end of class 
\emph{on paper}. This assignment is due at \emph{\due}.






\noindent Define a curve $\mathcal{C} = \begin{cases} x = \sign(t) \sqrt{|t|} \cos(|t|) \\ 
                                                                  y = \sign(t) \sqrt{|t|} \sin(|t|) 
                                             \end{cases} \, t \in [-3 \uppi, 3 \uppi]$.  The function $\sign$ is defined as
 $\sign(x) = \begin{cases} -1 & x < 0 \\  0 & x = 0 \\ 1 & x > 0 \end{cases}$.     
 Wikipedia tells me that this curve has been used for an ``efficient layout for the mirrors of concentrated solar power 
 plants.''
 
 You might like to use the Taylor series for the component functions centered at $\pi/2$; they are
 \begin{align*}
 x(t) &= \left( \frac{\sqrt{2} \sqrt{\ensuremath{\pi} } \left( t\operatorname{-}\frac{\ensuremath{\pi} }{2}\right) }{2}\right) \operatorname{-}\frac{\sqrt{2} \sqrt{\ensuremath{\pi} } {{\left( t\operatorname{-}\frac{\ensuremath{\pi} }{2}\right) }^{2}}}{2 \ensuremath{\pi} }\operatorname{+}\frac{\left( \sqrt{2} {{\ensuremath{\pi} }^{2}}\operatorname{+}3 \sqrt{2}\right)  {{\left( t\operatorname{-}\frac{\ensuremath{\pi} }{2}\right) }^{3}}}{12 \sqrt{\ensuremath{\pi} } \ensuremath{\pi} }\operatorname{+}\cdots, \\
y(t) &= \frac{\sqrt{2} \sqrt{\ensuremath{\pi} }}{2}\operatorname{+}\frac{\sqrt{2} \sqrt{\ensuremath{\pi} } \left( t\operatorname{-}\frac{\ensuremath{\pi} }{2}\right) }{2 \ensuremath{\pi} }\operatorname{-}\frac{\left( \sqrt{2} {{\ensuremath{\pi} }^{2}}\operatorname{+}\sqrt{2}\right)  {{\left( t\operatorname{-}\frac{\ensuremath{\pi} }{2}\right) }^{2}}}{4 \sqrt{\ensuremath{\pi} } \ensuremath{\pi} }\operatorname{-}\frac{\left( \sqrt{2} {{\ensuremath{\pi} }^{2}}\operatorname{-}\sqrt{2}\right)  {{\left( t\operatorname{-}\frac{\ensuremath{\pi} }{2}\right) }^{3}}}{4 \sqrt{\ensuremath{\pi} } {{\ensuremath{\pi} }^{2}}}\operatorname{+}\cdots.
\end{align*}                                      
                                          
\begin{questions} 
  
  \question[2] Ask Desmos to draw  $\mathcal{C}$. As best you can, reproduce the curve here.
  
  \begin{solution}%[2.5in]
  \end{solution}
  
  \newpage
  \question [2] Find $\displaystyle \left. \frac{\mathrm{d} y}{\mathrm{d} x} \right \vert_{t = \uppi/2}$
    \begin{solution}%[2.5in]
  \end{solution}
  
  \newpage
  
    \question [2] Find $\displaystyle \left. \frac{\mathrm{d}^2  y}{\mathrm{d} x^2} \right \vert_{t = \uppi/2}$
    \begin{solution}[2.5in]
  \end{solution}
  
  
  \end{questions}
  \end{document}