\documentclass[12pt,fleqn]{exam}
\usepackage{amssymb}
\usepackage[intlimits]{amsmath}
\usepackage{epsfig}
\usepackage{upgreek}
\usepackage[super]{nth}
\usepackage[colorlinks=true,linkcolor=black,anchorcolor=black,citecolor=black,filecolor=black,menucolor=black,runcolor=black,urlcolor=black]{hyperref}
\usepackage[letterpaper, margin=0.75in]{geometry}
\addpoints
\boxedpoints
\pointsinmargin
\pointname{pts}
\usepackage{tikz}
\usepackage{tkz-euclide}
\usetikzlibrary{shapes.geometric}
\usetikzlibrary{calc}
\usepackage[final]{microtype}
\frenchspacing
\usepackage[american]{babel}
\usepackage[T1]{fontenc}
\usepackage[]{fourier}

\usepackage{isomath}
\usepackage{upgreek,amsmath}
\usepackage{graphicx}

\newcommand{\dotprod}{\, {\scriptzcriptztyle\stackrel{\bullet}{{}}}\,}

\newcommand{\reals}{\mathbf{R}}
\newcommand{\lub}{\mathrm{lub}} 
\newcommand{\glb}{\mathrm{glb}} 
\newcommand{\complex}{\mathbf{C}}
\newcommand{\dom}{\mbox{dom}}
\newcommand{\range}{\mbox{range}}
\newcommand{\cover}{{\mathcal C}}
\newcommand{\integers}{\mathbf{Z}}
\newcommand{\vi}{\, \mathbf{i}}
\newcommand{\vj}{\, \mathbf{j}}
\newcommand{\vk}{\, \mathbf{k}}
\newcommand{\bi}{\, \mathbf{i}}
\newcommand{\bj}{\, \mathbf{j}}
\newcommand{\bk}{\, \mathbf{k}}
\DeclareMathOperator{\Arg}{\mathrm{Arg}}
\DeclareMathOperator{\Ln}{\mathrm{Ln}}
\newcommand{\imag}{\, \mathrm{i}}
\newcommand{\erf}{\mathrm{erf}}
\newcommand{\e}{\mathrm{e}}

\usepackage{graphicx}
\usepackage{color}
%\shadedsolutions
%\definecolor{SolutionColor}{rgb}{1,0.72,0.46} %{0.8,0.9,1}
\newcommand\AM{\textsc{am}}
\newcommand\PM{\textsc{pm}}


     
\newcommand{\quiz}{25}
\newcommand{\term}{Fall}
\newcommand{\due}{Thursday 2 May 13:20}
\newcommand{\class}{MATH 202, Spring \the\year}
\begin{document}
\large
\noindent\makebox[3.0truein][l]{\textbf{\class}}
\textbf{Name:} \hrulefill \\
\noindent \makebox[3.0truein][l]{\textbf{In class work  \quiz}}
\textbf{Row and Seat}:\hrulefill\\



\noindent  In class work  \textbf{\quiz}  has questions \textbf{1} 
through  \textbf{\numquestions} \/ with a total of 
\textbf{\numpoints\/} points. Turn in your work at the end of class 
\emph{on paper}. This assignment is due at \emph{\due}.

\vspace{0.1in}


\noindent \emph{“Perhaps my greatest wisdom is the knowledge that I do not know.”} \hfill 
{\sc John Steinbeck\footnote{ \emph{Travels with Charley: In Search of America }}}


\begin{questions} 
\question Define a curve $\mathcal{C}$ parametrically as $\mathcal{C} = \begin{cases} x = \cosh(t) \\ y = \sinh(t) \end{cases}, t \in \reals$.

\begin{parts}

\part [2] Ask Desmos to sketch $\mathcal{C}$ and reproduce the graph here.

\begin{solution}[3.0in]

\end{solution}

\part [2] Show that if $(x,y) \in \mathcal{C}$, then $x^2 - y^2 = 1$.
\begin{solution}[3.0in]

\end{solution}

\newpage
\part [2] Show that $(x=-1,y=0) \notin \mathcal{C}$, but that $(x=-1,y=0)$ is a point on the curve  $x^2 - y^2 = 1$.
\textbf{Hint:} Use the fact that $\range(\cosh) = [1,\infty)$.
\begin{solution}[3.0in]

\end{solution}


\part [2] Express the arclength of the portion of $\mathcal{C}$ if the parameter space is $[-1,1]$ as a definite integral. But do not attempt
to use the FTC to find this value (unless you want to learn about elliptic integrals).
\begin{solution}[3.0in]

\end{solution}

\end{parts}
    
   
\end{questions}
\end{document}
