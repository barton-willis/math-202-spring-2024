\documentclass[12pt,fleqn]{exam}
%\usepackage{pifont}
%\usepackage{dingbat,bbding}

\usepackage{amssymb}
\usepackage[intlimits]{amsmath}
\usepackage{epsfig}
\usepackage{upgreek}
\usepackage[super]{nth}
\usepackage[colorlinks=true,linkcolor=black,anchorcolor=black,citecolor=black,filecolor=black,menucolor=black,runcolor=black,urlcolor=black]{hyperref}
\usepackage[letterpaper, margin=0.75in]{geometry}
\addpoints
\boxedpoints
\pointsinmargin
\pointname{pts}
\usepackage{tikz}
\usepackage{tkz-euclide}
\usetikzlibrary{shapes.geometric}
\usetikzlibrary{calc}
\usepackage[final]{microtype}
\frenchspacing
\usepackage[american]{babel}
\usepackage[T1]{fontenc}
\usepackage[]{fourier}
\usepackage{isomath}
\usepackage{upgreek,amsmath}
\usepackage{amssymb}
\usepackage{graphicx}

\newcommand{\dotprod}{\, {\scriptzcriptztyle\stackrel{\bullet}{{}}}\,}

\newcommand{\reals}{\mathbf{R}}
\newcommand{\lub}{\mathrm{lub}} 
\newcommand{\glb}{\mathrm{glb}} 
\newcommand{\complex}{\mathbf{C}}
\newcommand{\dom}{\mbox{dom}}
\newcommand{\range}{\mbox{range}}
\newcommand{\cover}{{\mathcal C}}
\newcommand{\integers}{\mathbf{Z}}
\newcommand{\vi}{\, \mathbf{i}}
\newcommand{\vj}{\, \mathbf{j}}
\newcommand{\vk}{\, \mathbf{k}}
\newcommand{\bi}{\, \mathbf{i}}
\newcommand{\bj}{\, \mathbf{j}}
\newcommand{\bk}{\, \mathbf{k}}
\DeclareMathOperator{\Arg}{\mathrm{Arg}}
\DeclareMathOperator{\Ln}{\mathrm{Ln}}
\newcommand{\imag}{\, \mathrm{i}}

\usepackage{graphicx}
\usepackage{color}
%\shadedsolutions
%\definecolor{SolutionColor}{rgb}{1,0.72,0.46} %{0.8,0.9,1}
\newcommand\AM{\textsc{am}}
\newcommand\PM{\textsc{pm}}
     
\newcommand{\quiz}{18}
\newcommand{\term}{Spring}
\newcommand{\due}{Thursday 4 April   13:20}
\newcommand{\class}{MATH 202, Spring \the\year}
\begin{document}
\large
\vspace{0.1in}
\noindent\makebox[3.0truein][l]{\textbf{\class}}
\textbf{Name:} \hrulefill \\
\noindent \makebox[3.0truein][l]{\textbf{In class work  \quiz}}
\textbf{Row and Seat}:\hrulefill\\

\vspace{0.1in}
\noindent  In class work  \textbf{\quiz\/}  has questions \textbf{1} through  \textbf{\numquestions} \/ with a total of \textbf{\numpoints\/}  points.   

\vspace{0.1in}

\noindent  \emph{
“Folks are usually about as happy as they make their minds up to be.” } \\ $\phantom{x}$ \hfill \textsc{ Abraham Lincoln }


Fun facts to know and tell:  For all $k \in \integers_{\geq 0}$, we have
\begin{align*}
    (k + 1)! &= (k+1) k!\\
    (k+2)! &= (k+1)(k+2) k!\\
    (2 k + 1)! &= (2k + 1) (2 k)! \\
        (2 k + 2)! &= (2k + 1) (2 k + 1)  (2 k)!
\end{align*}


\begin{questions}

\question Use the RT to determine all real numbers $x$ that make the given series converge absolutely.

\begin{parts}

\part [2] $\displaystyle \sum_{k=1}^\infty \frac{x^k}{\sqrt{k}}$

\begin{solution}%[3.0in]

\end{solution}

\newpage

\part [2] $\displaystyle \sum_{k=0}^\infty  \frac{(k!)^2}{ (2k)!} x^k$

\begin{solution}%[4.5in]

\end{solution}

\newpage

\part [2]  $\displaystyle \sum_{k=0}^\infty  \frac{(2x+1)^k}{k^2+1}$


\end{parts}
\end{questions}
\end{document}

