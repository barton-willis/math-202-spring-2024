\documentclass[12pt,fleqn]{exam}
\usepackage{amssymb}
\usepackage[intlimits]{amsmath}
\usepackage{epsfig}
\usepackage{upgreek}
\usepackage[super]{nth}
\usepackage[colorlinks=true,linkcolor=black,anchorcolor=black,citecolor=black,filecolor=black,menucolor=black,runcolor=black,urlcolor=black]{hyperref}
\usepackage[letterpaper, margin=0.75in]{geometry}
\addpoints
\boxedpoints
\pointsinmargin
\pointname{pts}
\usepackage{tikz}
\usepackage{tkz-euclide}
\usetikzlibrary{shapes.geometric}
\usetikzlibrary{calc}
\usepackage[final]{microtype}
\frenchspacing
\usepackage[american]{babel}
\usepackage[T1]{fontenc}
\usepackage[]{fourier}
\usepackage{isomath}
\usepackage{upgreek,amsmath}
\usepackage{graphicx}

\newcommand{\dotprod}{\, {\scriptzcriptztyle\stackrel{\bullet}{{}}}\,}

\newcommand{\reals}{\mathbf{R}}
\newcommand{\lub}{\mathrm{lub}} 
\newcommand{\glb}{\mathrm{glb}} 
\newcommand{\complex}{\mathbf{C}}
\newcommand{\dom}{\mbox{dom}}
\newcommand{\range}{\mbox{range}}
\newcommand{\cover}{{\mathcal C}}
\newcommand{\integers}{\mathbf{Z}}
\newcommand{\vi}{\, \mathbf{i}}
\newcommand{\vj}{\, \mathbf{j}}
\newcommand{\vk}{\, \mathbf{k}}
\newcommand{\bi}{\, \mathbf{i}}
\newcommand{\bj}{\, \mathbf{j}}
\newcommand{\bk}{\, \mathbf{k}}
\DeclareMathOperator{\Arg}{\mathrm{Arg}}
\DeclareMathOperator{\Ln}{\mathrm{Ln}}
\newcommand{\imag}{\, \mathrm{i}}

\usepackage{graphicx}
\usepackage{color}
%\shadedsolutions
%\definecolor{SolutionColor}{rgb}{1,0.72,0.46} %{0.8,0.9,1}
\newcommand\AM{\textsc{am}}
\newcommand\PM{\textsc{pm}}
     
\newcommand{\quiz}{17}
\newcommand{\term}{Fall}
\newcommand{\due}{Tuesday 9 April 13:20}
\newcommand{\class}{MATH 202, Fall \the\year}
\begin{document}
%\large
\vspace{0.1in}
\noindent\makebox[3.0truein][l]{\textbf{\class}}
\textbf{Name:} \hrulefill \\
\noindent \makebox[3.0truein][l]{\textbf{In class work  \quiz}}
\textbf{Row and Seat}:\hrulefill\\

\noindent \emph{“The more I read, the more I acquire, the more certain I am that I know nothing.”} \hfill {\sc Voltaire }

\vspace{0.1in}

\noindent  In class work  \textbf{\quiz}  has questions \textbf{1} 
through  \textbf{\numquestions} \/ with a total of 
\textbf{\numpoints\/} points. Turn in your work at the end of class 
\emph{on paper}. This assignment is due \emph{\due}.

\vspace{0.1in}



\begin{questions} 

\question [1] Define a sequence $s$ by $\displaystyle s_n = \sum_{k=0}^n \frac{(-1)^{k}}{(k+1)^{3/2}}$.  This is a convergent alternating series.
Also define \mbox{$\displaystyle s_\infty = \lim_{n \to \infty}  s_n$.}

\begin{parts}

\part [1] Use Desmos to graph $s$ on the interval $[1, 2, \dots, 300]$. Also use Desmos to find the numeric values of $s_{299}$ and  $s_{300}$.
As best you can, reproduce a cartoon of the graph of $s$.

\begin{solution}%[3.6in]
\end{solution}

\newpage
\part[1] From the theory of convergent alternating series, we know that $s_{299} < s_\infty < s_{300}$.   From this, we can conclude that
\begin{equation*}
      s_\infty = \left( \frac{s_{300} + s_{299}}{2} \right) \pm \left( \frac{s_{300} - s_{299}}{2} \right)
\end{equation*}
Here we are using the $\pm$ symbol to mean some number in a interval; specifically for $a \in \reals$ and $b \in \reals_{>0}$, by  $a \pm b$ we mean some number in
the closed interval $[a-b, a+ b]$. Express $s_\infty$ in the form of $a \pm b$.

\begin{solution}[3.6in]
\end{solution}



\end{parts}


\end{questions}
\end{document}
