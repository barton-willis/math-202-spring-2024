\documentclass[12pt,fleqn,answers]{exam}
%\usepackage{pifont}
%\usepackage{dingbat,bbding}

\usepackage{amssymb}
\usepackage[intlimits]{amsmath}
\usepackage{epsfig}
\usepackage{upgreek}
\usepackage[super]{nth}
\usepackage[colorlinks=true,linkcolor=black,anchorcolor=black,citecolor=black,filecolor=black,menucolor=black,runcolor=black,urlcolor=black]{hyperref}
\usepackage[letterpaper, margin=0.75in]{geometry}
\addpoints
\boxedpoints
\pointsinmargin
\pointname{pts}
\usepackage{tikz}
\usepackage{tkz-euclide}
\usetikzlibrary{shapes.geometric}
\usetikzlibrary{calc}
\usepackage[final]{microtype}
\frenchspacing
\usepackage[american]{babel}
\usepackage[T1]{fontenc}
\usepackage[]{fourier}
\usepackage{isomath}
\usepackage{upgreek,amsmath}
\usepackage{amssymb}
\usepackage{graphicx}

\newcommand{\dotprod}{\, {\scriptzcriptztyle\stackrel{\bullet}{{}}}\,}

\newcommand{\reals}{\mathbf{R}}
\newcommand{\lub}{\mathrm{lub}} 
\newcommand{\glb}{\mathrm{glb}} 
\newcommand{\complex}{\mathbf{C}}
\newcommand{\dom}{\mbox{dom}}
\newcommand{\range}{\mbox{range}}
\newcommand{\cover}{{\mathcal C}}
\newcommand{\integers}{\mathbf{Z}}
\newcommand{\vi}{\, \mathbf{i}}
\newcommand{\vj}{\, \mathbf{j}}
\newcommand{\vk}{\, \mathbf{k}}
\newcommand{\bi}{\, \mathbf{i}}
\newcommand{\bj}{\, \mathbf{j}}
\newcommand{\bk}{\, \mathbf{k}}
\DeclareMathOperator{\Arg}{\mathrm{Arg}}
\DeclareMathOperator{\Ln}{\mathrm{Ln}}
\newcommand{\imag}{\, \mathrm{i}}

\usepackage{graphicx}
\usepackage{xcolor}
\shadedsolutions
%\definecolor{shadecolor}{RGB}{200,200,200}
\colorlet{shadecolor}{gray!10}
%\colorlet{shadecolor}{gray!40}
\newcommand\AM{\textsc{am}}
\newcommand\PM{\textsc{pm}}
     
\newcommand{\quiz}{3}
\newcommand{\term}{Spring }
\newcommand{\due}{Thursday 1 February 13:20}
\newcommand{\class}{MATH 202, \term \the\year}
\begin{document}
\large
\vspace{0.1in}
\noindent\makebox[3.0truein][l]{\textbf{\class}}
\textbf{Name:} \hrulefill \\
\noindent \makebox[3.0truein][l]{\textbf{In class work \quiz}}
\textbf{Row and Seat}:\hrulefill\\
\vspace{0.1in}


\noindent  In class work  \textbf{\quiz\/}  has questions \textbf{1} through  \textbf{\numquestions} \/ with a total of \textbf{\numpoints\/}  points.   
Turn in your work at the end of class  \emph{on paper}. This assignment is due \emph{\due}.

\vspace{0.1in}


\begin{questions} 

    \question[2] Evaluate the definite integral 
    $\int_0^2  x  \sqrt{1 + x^2} \, \mathrm{d} x$ by using 
    the substitution $z = 1 + x^2$.  
    \begin{solution}[4.5in] \textbf{Method I} 
    
    Let $z = 1+x^2$. Then $\mathrm{d} z = 2 x \mathrm{d} x$. The integrand has a factor of $ x \mathrm{d} x$, so
    let's solve  $\mathrm{d} z = 2 x \mathrm{d} x$ for $ x \mathrm{d} x$. Thus $ x \mathrm{d} x = \frac{1}{2} \mathrm{d} z$.  When $x = 0$, we have $z = 1$; when
    $x=2$, we have $z = 5$. So
    \begin{align}
     \int_0^2  x  \sqrt{1 + x^2} \, \mathrm{d} x &= \int_1^5 \frac{1}{2} \sqrt{z} \, \mathrm{d} z,  && (\text{substitution}) \\
            &= \left. \frac{1}{2} \times \frac{2}{3} z^{3/2} \right \vert_{z=1}^{z=5},   && (\text{FTC}) \\
            &= \frac{1}{3} \left( 5^{3/2} - 1 \right).  && (\text{algebra}) 
    \end{align}
    
    \textbf{Method II}  For Method II, we first find an anti derivative by using a substitution, second we return the 
    original variable, and finally we use the FTC.   Let $z = 1+x^2$. Then $\mathrm{d} z = 2 x \mathrm{d} x$. The integrand has a factor of $ x \mathrm{d} x$, so
    let's solve  $\mathrm{d} z = 2 x \mathrm{d} x$ for $ x \mathrm{d} x$. Thus $ x \mathrm{d} x = \frac{1}{2} \mathrm{d} z$.   So
    
\begin{align}
     \int x  \sqrt{1 + x^2} \, \mathrm{d} x &= \int \frac{1}{2} \sqrt{z} \, \mathrm{d} z,  && (\text{substitution}) \\
                                                             &=  \frac{1}{2} \times \frac{1}{3} z^{3/2},   
                                                              && (\text{power rule for antiderivatives }) \\
                                                             &= \frac{1}{3}  (1+x^2)^{3/2}.  && (\text{return to original variable}) 
    \end{align}
    
  Now we can use the FTC
 \begin{align}
     \int_0^2  x  \sqrt{1 + x^2} \, \mathrm{d} x &= \left.  \frac{1}{3}  (1+x^2)^{3/2} \right \vert_{x=0}^{x=2},\\
                                                                     &= \frac{1}{3} \left( 5^{3/2} - 1 \right).
\end{align}
        \end{solution}

    \newpage

    \question The force required to extend a spring is proportional to 
    the amount of extension.

    \begin{parts}

    \part [2]  If a force of 10 Newtons extends the spring
     $0.03$ meters, find the formula for the force $F$ required to 
     extend the spring $x$ meters.

     \begin{solution}[4.5in] The force $F$ as a function of displacement $x$ has the form $F = kx$, where $k$ is a number.
     From the given data, we have $10 = k \times 0.03$.  So $k = \frac{1000}{3} \,\,  \mathrm{N}/\mathrm{m}$.
    

    
        
     \end{solution}

     \part[2] Find the \emph{work} required to extend the spring $0.05$ meters.
      If you don't know, the MKS unit of work is the Joule (which is Newton $\times$ meter).
      (To maybe make this more tangible: To lift a two pound sack of sugar three feet up, the work required is
      about one Joule.)
      
      \begin{solution}[4.5in] 
      \begin{equation}
        W = \int_0^{0.05}  \frac{1000}{3} x \, \mathrm{d} x =  \left. \frac{1000}{6} x^2 \right |_{x=0}^{x = 0.05}
        = \frac{5}{12} \,\, \mathrm{J}.
      \end{equation}
    
        
     \end{solution}

    \end{parts}


  
\end{questions}

\end{document}

