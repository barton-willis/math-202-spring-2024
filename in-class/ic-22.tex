\documentclass[12pt,fleqn]{exam}
\usepackage{amssymb}
\usepackage[intlimits]{amsmath}
\usepackage{epsfig}
\usepackage{upgreek}
\usepackage[super]{nth}
\usepackage[colorlinks=true,linkcolor=black,anchorcolor=black,citecolor=black,filecolor=black,menucolor=black,runcolor=black,urlcolor=black]{hyperref}
\usepackage[letterpaper, margin=0.75in]{geometry}
\addpoints
\boxedpoints
\pointsinmargin
\pointname{pts}
\usepackage{tikz}
\usepackage{tkz-euclide}
\usetikzlibrary{shapes.geometric}
\usetikzlibrary{calc}
\usepackage[final]{microtype}
\frenchspacing
\usepackage[american]{babel}
\usepackage[T1]{fontenc}
\usepackage[]{fourier}
\usepackage{isomath}
\usepackage{upgreek,amsmath}
\usepackage{graphicx}

\newcommand{\dotprod}{\, {\scriptzcriptztyle\stackrel{\bullet}{{}}}\,}

\newcommand{\reals}{\mathbf{R}}
\newcommand{\lub}{\mathrm{lub}} 
\newcommand{\glb}{\mathrm{glb}} 
\newcommand{\complex}{\mathbf{C}}
\newcommand{\dom}{\mbox{dom}}
\newcommand{\range}{\mbox{range}}
\newcommand{\cover}{{\mathcal C}}
\newcommand{\integers}{\mathbf{Z}}
\newcommand{\vi}{\, \mathbf{i}}
\newcommand{\vj}{\, \mathbf{j}}
\newcommand{\vk}{\, \mathbf{k}}
\newcommand{\bi}{\, \mathbf{i}}
\newcommand{\bj}{\, \mathbf{j}}
\newcommand{\bk}{\, \mathbf{k}}
\DeclareMathOperator{\Arg}{\mathrm{Arg}}
\DeclareMathOperator{\Ln}{\mathrm{Ln}}
\newcommand{\imag}{\, \mathrm{i}}

\usepackage{graphicx}
\usepackage{color}
%\shadedsolutions
%\definecolor{SolutionColor}{rgb}{1,0.72,0.46} %{0.8,0.9,1}
\newcommand\AM{\textsc{am}}
\newcommand\PM{\textsc{pm}}
     
\newcommand{\quiz}{22}
\newcommand{\term}{Fall}
\newcommand{\due}{Thursday 11 April 13:20}
\newcommand{\class}{MATH 202, Fall \the\year}
\begin{document}
%\large
\vspace{0.1in}
\noindent\makebox[3.0truein][l]{\textbf{\class}}
\textbf{Name:} \hrulefill \\
\noindent \makebox[3.0truein][l]{\textbf{In class work  \quiz}}
\textbf{Row and Seat}:\hrulefill\\

\noindent \emph{“I'm killing time while I wait for life to shower me with meaning and happiness.” } \hfill {\sc Calvin}

\vspace{0.1in}

\noindent  In class work  has questions \textbf{1} 
through  \textbf{\numquestions} \/ with a total of 
\textbf{\numpoints\/} points. Turn in your work at the end of class 
\emph{on paper}. This assignment is due \emph{\due}.

\vspace{0.1in}


\noindent On family math night, my friends Bea and Ernest Kind used the ratio test to prove that the power series $\sum_{k=0}^\infty \frac{x^k}{k!}$ converges for
all real numbers $x$.  In honor of their teenager  Emma Kind, Bea and Ernest define a function $E(x) = \sum_{k=0}^\infty \frac{x^k}{k!}$, where the domain of $E$ is all real numbers.

\begin{questions} 

\question[2] Find the numerical value of $E(0)$; find the numerical value of $E^\prime(0)$.To find these values, go to your happy place: we have
$E(x) = 1 + x + \frac{x^2}{2} +  \frac{x^3}{6} +  \frac{x^4}{24} + \cdots.$
\begin{solution}[1.5in]
\end{solution}
\question[2] Bea and Ernest are curious about the graph of the equation $y = E(x)$. They realize that summing to infinity isn't really an option, so they ask their favorite graphing tool to
graph $y = \sum_{k=0}^{42} \frac{x^k}{k!}$. They reason that for since the factorial grows very quickly that sum of the first 43 terms will be a good approximation to $E$, at least for
modest values of $x$.      Ask your favorite graphing tool to graph $y = \sum_{k=0}^{42} \frac{x^k}{k!}$. Does the graph look like a function that has a standard name? Reproduce
the graph here:
\begin{solution}[3.5in]
\end{solution}

\newpage
\question[2]  Find the power series for $E^\prime$. Change the sum index so that the least sum index is one.   Is there a conspiracy between $E(x)$ and $E^\prime(x)$? If so, what is it?
\begin{solution}[3.5in]
\end{solution}


\question[2] What is the standard name for the function $E$?
\begin{solution}[3.5in]
\end{solution}





\end{questions}
\end{document}
