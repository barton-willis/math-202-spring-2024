\documentclass[12pt,fleqn]{exam}
%\usepackage{pifont}
%\usepackage{dingbat,bbding}

\usepackage{amssymb}
\usepackage[intlimits]{amsmath}
\usepackage{epsfig}
\usepackage{upgreek}
\usepackage[super]{nth}
\usepackage[colorlinks=true,linkcolor=black,anchorcolor=black,citecolor=black,filecolor=black,menucolor=black,runcolor=black,urlcolor=black]{hyperref}
\usepackage[letterpaper, margin=0.75in]{geometry}
\addpoints
\boxedpoints
\pointsinmargin
\pointname{pts}
\usepackage{tikz}
\usepackage{tkz-euclide}
\usetikzlibrary{shapes.geometric}
\usetikzlibrary{calc}
\usepackage[final]{microtype}
\frenchspacing
\usepackage[american]{babel}
\usepackage[T1]{fontenc}
\usepackage[]{fourier}
\usepackage{isomath}
\usepackage{upgreek,amsmath}
\usepackage{amssymb}
\usepackage{graphicx}

\newcommand{\dotprod}{\, {\scriptzcriptztyle\stackrel{\bullet}{{}}}\,}

\newcommand{\reals}{\mathbf{R}}
\newcommand{\lub}{\mathrm{lub}} 
\newcommand{\glb}{\mathrm{glb}} 
\newcommand{\complex}{\mathbf{C}}
\newcommand{\dom}{\mbox{dom}}
\newcommand{\range}{\mbox{range}}
\newcommand{\cover}{{\mathcal C}}
\newcommand{\integers}{\mathbf{Z}}
\newcommand{\vi}{\, \mathbf{i}}
\newcommand{\vj}{\, \mathbf{j}}
\newcommand{\vk}{\, \mathbf{k}}
\newcommand{\bi}{\, \mathbf{i}}
\newcommand{\bj}{\, \mathbf{j}}
\newcommand{\bk}{\, \mathbf{k}}
\DeclareMathOperator{\Arg}{\mathrm{Arg}}
\DeclareMathOperator{\Ln}{\mathrm{Ln}}
\newcommand{\imag}{\, \mathrm{i}}

\usepackage{graphicx}
\usepackage{color}
%\shadedsolutions
%\definecolor{SolutionColor}{rgb}{1,0.72,0.46} %{0.8,0.9,1}
\newcommand\AM{\textsc{am}}
\newcommand\PM{\textsc{pm}}
     
\newcommand{\quiz}{12}
\newcommand{\term}{Spring}
\newcommand{\due}{Tuesday 5 March  13:20}
\newcommand{\class}{MATH 202, Spring \the\year}
\begin{document}
\large
\vspace{0.1in}
\noindent\makebox[3.0truein][l]{\textbf{\class}}
\textbf{Name:} \hrulefill \\
\noindent \makebox[3.0truein][l]{\textbf{In class work  \quiz}}
\textbf{Row and Seat}:\hrulefill\\
\vspace{0.1in}
\vspace{0.1in}
\noindent  In class work  \textbf{\quiz\/}  has questions \textbf{1} through  \textbf{\numquestions} \/ with a total of \textbf{\numpoints\/}  points.   

\vspace{0.1in}
\noindent 

\emph{``The best teacher is experience and not through someone's distorted point of view.”} \\
$\phantom{xxx}$  \hfill \textsc{Jack Kerouac}, \emph{On the Road}

\begin{questions}

 \question Find each antiderivative:

 \begin{parts}

    \part [2] $\int \frac{x+1}{(x-3)(x-9)} \, \mathrm{d}x $

\newpage

    \begin{solution}%[2.5in]
         \begin{equation*}
        \int \frac{x+1}{(x+3)(x-9)} \, \mathrm{d} x 
        = \int \frac{1}{6 \left( x+3\right) }+\frac{5}{6 \left( x-9\right) }
        \, \mathrm{d} x =  \frac{\log{\left( \left| x+3\right| \right) }}{6}+\frac{5 \log{\left( \left| x-9\right| \right) }}{6}
        \end{equation*}
    \end{solution}

    \part [2] $\int \frac{1}{{{\left( x-1\right) }^{2}} x} 
     \, \mathrm{d}x$
     \begin{solution}[2.5in]
        \begin{equation*}
            \int \frac{1}{{{\left( x-1\right) }^{2}} x} \, \mathrm{d} x 
            = \int \frac{1}{x}-\frac{1}{x-1}+\frac{1}{{{\left( x-1\right) }^{2}}}
             \, \mathrm{d} x = 
             \ln{\left( \left| x\right| \right) }-\frac{1}{x-1}-\ln{\left( \left| x-1\right| \right) }
        \end{equation*}
     \end{solution}
  
  \end{parts}
  \end{questions}
  \end{document}
  
     \newpage 

     \part [2] $\int \frac{1}{x^2 + 18 x +1} \, \mathrm{d}x$ 
     
     \textbf{Hint:} The factorization of $x^2 + 18 x +1$ is
     the gnarly $\left( x-4 \sqrt{5}+9\right) \, \left( x+4 \sqrt{5}+9\right)$.
     So the pfd has the form
     \begin{equation*}
        \frac{1}{x^2 + 18 x +1} = \frac{A}{x-4 \sqrt{5}+9} 
          + \frac{B}{ x+4 \sqrt{5}+9}
     \end{equation*}
     And sure, you can solve the problem starting with this. An alternative
     is to make a substitution  $x = z + \alpha$, where $\alpha$
     is a  ``magic'' number you choose to transform the 
     problem into comfort math. And comfort math might be
     \begin{equation*}
         \frac{\mathrm{d}}{\mathrm{d} x} \left (\frac{1}{\alpha}
          \tanh^{-1} \left(\frac{x}{\alpha} \right) \right)
          = \frac{1}{{{\alpha }^{2}}-{{x}^{2}}}.
     \end{equation*} 
     
   \begin{solution}
    \begin{align*}
    \int \frac{1}{x^2 + 18 x +1} \, \mathrm{d} x
    &= \int \frac{1}{(z+\alpha)^2 + 18 (z+\alpha) + 1} \, \mathrm{d} z \\
    &= \int \frac{1}{z^2 + (18 + 2 \alpha) + 18 \alpha +1} \,
    \mathrm{d} z, \\
    \intertext{Choose $\alpha = -9$. Then} 
    &= \int \frac{1}{z^2 - 80} \,   \mathrm{d} z, \\
    &= -\frac{1}{\sqrt{80}} \tanh^{-1} \left(z / \sqrt{80}\right), \\
    &=  -\frac{1}{\sqrt{80}} \tanh^{-1} \left((x+9) / \sqrt{80}\right)
    \end{align*}
   \end{solution}
 \end{parts}
\end{questions}
\end{document}

