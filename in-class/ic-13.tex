\documentclass[12pt,fleqn]{exam}
%\usepackage{pifont}
%\usepackage{dingbat,bbding}

\usepackage{amssymb}
\usepackage[intlimits]{amsmath}
\usepackage{epsfig}
\usepackage{upgreek}
\usepackage[super]{nth}
\usepackage[colorlinks=true,linkcolor=black,anchorcolor=black,citecolor=black,filecolor=black,menucolor=black,runcolor=black,urlcolor=black]{hyperref}
\usepackage[letterpaper, margin=0.75in]{geometry}
\addpoints
\boxedpoints
\pointsinmargin
\pointname{pts}
\usepackage{tikz}
\usepackage{tkz-euclide}
\usetikzlibrary{shapes.geometric}
\usetikzlibrary{calc}
\usepackage[final]{microtype}
\frenchspacing
\usepackage[american]{babel}
\usepackage[T1]{fontenc}
\usepackage[]{fourier}
\usepackage{isomath}
\usepackage{upgreek,amsmath}
\usepackage{amssymb}
\usepackage{graphicx}

\newcommand{\dotprod}{\, {\scriptzcriptztyle\stackrel{\bullet}{{}}}\,}

\newcommand{\reals}{\mathbf{R}}
\newcommand{\lub}{\mathrm{lub}} 
\newcommand{\glb}{\mathrm{glb}} 
\newcommand{\complex}{\mathbf{C}}
\newcommand{\dom}{\mbox{dom}}
\newcommand{\range}{\mbox{range}}
\newcommand{\cover}{{\mathcal C}}
\newcommand{\integers}{\mathbf{Z}}
\newcommand{\vi}{\, \mathbf{i}}
\newcommand{\vj}{\, \mathbf{j}}
\newcommand{\vk}{\, \mathbf{k}}
\newcommand{\bi}{\, \mathbf{i}}
\newcommand{\bj}{\, \mathbf{j}}
\newcommand{\bk}{\, \mathbf{k}}
\DeclareMathOperator{\Arg}{\mathrm{Arg}}
\DeclareMathOperator{\Ln}{\mathrm{Ln}}
\newcommand{\imag}{\, \mathrm{i}}

\usepackage{graphicx}
\usepackage{color}
%\shadedsolutions
%\definecolor{SolutionColor}{rgb}{1,0.72,0.46} %{0.8,0.9,1}
\newcommand\AM{\textsc{am}}
\newcommand\PM{\textsc{pm}}
     
\newcommand{\quiz}{13}
\newcommand{\term}{Spring}
\newcommand{\due}{Tuesday 5 March  13:20}
\newcommand{\class}{MATH 202, Spring \the\year}
\begin{document}
\large
\vspace{0.1in}
\noindent\makebox[3.0truein][l]{\textbf{\class}}
\textbf{Name:} \hrulefill \\
\noindent \makebox[3.0truein][l]{\textbf{In class work  \quiz}}
\textbf{Row and Seat}:\hrulefill\\
\vspace{0.1in}
\vspace{0.1in}
\noindent  In class work  \textbf{\quiz\/}  has questions \textbf{1} through  \textbf{\numquestions} \/ with a total of \textbf{\numpoints\/}  points.   

\vspace{0.1in}
\noindent  \emph{
“Great things are not accomplished by those who yield to trends and fads and popular opinion.”} 
\hfill \textsc{Jack Kerouac}

\noindent Here is a copy of Larry's top secret short table of obscure integrals (STOI)
\begin{align*}
&\int {\left. \left| x\right| \, \mathrm{d}x\right.} =\frac{x\, \left| x\right| }{2}, &&\int {\left. x\, \left| x\right|  \, \mathrm{d} x\right.} =\frac{{{x}^{2}}\, \left| x\right| }{3}, \\
&\int {\left. {{x}^{2}}\, \left| x\right|  \, \mathrm{d} x\right.}=\frac{{{x}^{3}}\, \left| x\right| }{4}, &&\int \lfloor x \rfloor \, \mathrm{d} x =  - \frac{1}{2} 
 \lfloor x \rfloor \left(\lfloor x \rfloor - 2 x + 1\right). 
\end{align*}
\begin{questions}

\question [1] Use  seventh grade geometry to find the numerical value of $\int_{-2}^3  |z| \, \mathrm{d} z$.
\begin{solution}[2.5in]

\end{solution}

\question [1] Use  the STOI to find the numerical value of $\int_{-2}^3  |z| \, \mathrm{d} z$.
\begin{solution}%[3.5in]

\end{solution}

\newpage

\question [1] Use seventh grade geometry to find the numerical value of $\int_1^5  \lfloor z \rfloor \, \mathrm{d} z$.
\begin{solution}[3.5in]

\end{solution}

\question [1] Use the STOI to find the numerical value of $\int_1^5  \lfloor z \rfloor \, \mathrm{d} z$.
\begin{solution}%[3.5in]

\end{solution}

\newpage

 \question [1] According to the STOI, we have $\int \lfloor x \rfloor \, \mathrm{d} x = - \frac{1}{2} 
 \lfloor x \rfloor \left(\lfloor x \rfloor - 2 x + 1\right)$. Ask Desmos to graph
 $y = - \frac{1}{2}  \lfloor x \rfloor \left(\lfloor x \rfloor - 2 x + 1\right)$.  Draw the graph here. 
Does the graph appear to be continuous?


\begin{solution}[3.5in]

\end{solution}

\question [1] Use the STOI to find the numerical value of $\int_0^\uppi 2 \left \lfloor \frac{x}{2} \right \rfloor \, \mathrm{d} x$.
\textbf{Hint:} Substitute $z = x/2$.
\end{questions}
\end{document}

