\documentclass[12pt,fleqn]{exam}
%\usepackage{pifont}
%\usepackage{dingbat,bbding}

\usepackage{amssymb}
\usepackage[intlimits]{amsmath}
\usepackage{epsfig}
\usepackage{upgreek}
\usepackage[super]{nth}
\usepackage[colorlinks=true,linkcolor=black,anchorcolor=black,citecolor=black,filecolor=black,menucolor=black,runcolor=black,urlcolor=black]{hyperref}
\usepackage[letterpaper, margin=0.75in]{geometry}
\addpoints
\boxedpoints
\pointsinmargin
\pointname{pts}
\usepackage{tikz}
\usepackage{tkz-euclide}
\usetikzlibrary{shapes.geometric}
\usetikzlibrary{calc}
\usepackage[final]{microtype}
\frenchspacing
\usepackage[american]{babel}
\usepackage[T1]{fontenc}
\usepackage[]{fourier}
\usepackage{isomath}
\usepackage{upgreek,amsmath}
\usepackage{amssymb}
\usepackage{graphicx}

\newcommand{\dotprod}{\, {\scriptzcriptztyle\stackrel{\bullet}{{}}}\,}

\newcommand{\reals}{\mathbf{R}}
\newcommand{\lub}{\mathrm{lub}} 
\newcommand{\glb}{\mathrm{glb}} 
\newcommand{\complex}{\mathbf{C}}
\newcommand{\dom}{\mbox{dom}}
\newcommand{\range}{\mbox{range}}
\newcommand{\cover}{{\mathcal C}}
\newcommand{\integers}{\mathbf{Z}}
\newcommand{\vi}{\, \mathbf{i}}
\newcommand{\vj}{\, \mathbf{j}}
\newcommand{\vk}{\, \mathbf{k}}
\newcommand{\bi}{\, \mathbf{i}}
\newcommand{\bj}{\, \mathbf{j}}
\newcommand{\bk}{\, \mathbf{k}}
\DeclareMathOperator{\Arg}{\mathrm{Arg}}
\DeclareMathOperator{\Ln}{\mathrm{Ln}}
\newcommand{\imag}{\, \mathrm{i}}

\newtheorem{theorem}{Theorem}
\usepackage{graphicx}
\usepackage{color}
%\shadedsolutions
%\definecolor{SolutionColor}{rgb}{1,0.72,0.46} %{0.8,0.9,1}
\newcommand\AM{\textsc{am}}
\newcommand\PM{\textsc{pm}}
     
\newcommand{\quiz}{13}
\newcommand{\term}{Spring}
\newcommand{\due}{Tuesday 19 March  13:20}
\newcommand{\class}{MATH 202, Spring \the\year}
\begin{document}
\large
\vspace{0.1in}
\noindent\makebox[3.0truein][l]{\textbf{\class}}
\textbf{Name:} \hrulefill \\
\noindent \makebox[3.0truein][l]{\textbf{In class work  \quiz}}
\textbf{Row and Seat}:\hrulefill\\
\vspace{0.1in}
\vspace{0.1in}
\noindent  In class work  \textbf{\quiz\/}  has questions \textbf{1} through  \textbf{\numquestions} \/ with a total of \textbf{\numpoints\/}  points.   

\vspace{0.1in}

\noindent  \emph{``A linguistic construction is called referentially transparent when for any expression built from it, replacing a subexpression with another one that denotes the same value does not change the value of the expression.''}  \hfill  (Wikipedia)

\begin{theorem}[CT] Let $a$ and $b$ be seqences. And suppose that $\left(\forall k \in \integers_{\geq 0}\right)(0 \leq a_k \leq b_k)$. Then 
$\sum b \text{ converges } \implies \sum a  \text{ converges}.$
\end{theorem}


\begin{theorem}[LCT] Let $a$ and $b$ be seqences. And suppose that there is and integer $N$ such that for all $k \in \integers_{\geq N}$, we have
$0 < a_k$ and $0 < b_k$. Then
\begin{itemize}
 \item $\displaystyle \lim_{k \to \infty} \left(\frac{a_k}{b_k} \right) \in \reals_{>0} \implies  \left( \sum a  \text{ converges} \equiv \sum b  \text{ converges} \right) $
 \item $\displaystyle \lim_{k \to \infty} \left( \frac{a_k}{b_k} \right) = 0 \implies  \left( \sum b  \text{ converges} \implies \sum a  \text{ converges} \right) $
  \item $\displaystyle \lim_{k \to \infty} \left(\frac{a_k}{b_k} \right) = \infty  \implies  \left( \sum a  \text{ converges} \implies \sum b  \text{ converges} \right) $
 
 
\end{itemize}


\end{theorem}
\begin{questions}

\question [2] Use the CT to show that $\sum_{k=0}^\infty  \frac{1}{k^2 + k + 28}$ converges.

\newpage


\question [2] Use the LCT to show that $\sum_{k=0}^\infty \begin{cases} (-1)^k k! & k < 10^9 \\   \frac{1}{k^8 + 1} & k \geq 10^9 \end{cases}$ converges.



\newpage 

\question [2] After an arduous calcuation spanning tweleve pages of engineering paper, my friend Lilly Poole has (correctly) shown that
\begin{equation*}
\int_{0}^{\infty }{\left. \frac{1}{{{x}^{8}}\operatorname{+}1} \, \mathrm{d} x\right.}\operatorname{=}\frac{\ensuremath{\uppi} }{8 \sin{\left( \frac{\ensuremath{\uppi} }{8}\right) }}.
\end{equation*}
After that, Ms.\ Poole (spouse of Mr.\  Wade Poole) cocludes that $\sum_{k=0}^\infty \frac{1}{1+k^8} = \frac{\ensuremath{\uppi} }{8 \sin{\left( \frac{\ensuremath{\uppi} }{8}\right) }}$.
Is  Ms.\ Poole's conclusion correct? Explain.

\end{questions}
\end{document}