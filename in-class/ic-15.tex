\documentclass[12pt,fleqn]{exam}
%\usepackage{pifont}
%\usepackage{dingbat,bbding}

\usepackage{amssymb}
\usepackage[intlimits]{amsmath}
\usepackage{epsfig}
\usepackage{upgreek}
\usepackage[super]{nth}
\usepackage[colorlinks=true,linkcolor=black,anchorcolor=black,citecolor=black,filecolor=black,menucolor=black,runcolor=black,urlcolor=black]{hyperref}
\usepackage[letterpaper, margin=0.75in]{geometry}
\addpoints
\boxedpoints
\pointsinmargin
\pointname{pts}
\usepackage{tikz}
\usepackage{tkz-euclide}
\usetikzlibrary{shapes.geometric}
\usetikzlibrary{calc}
\usepackage[final]{microtype}
\frenchspacing
\usepackage[american]{babel}
\usepackage[T1]{fontenc}
\usepackage[]{fourier}
\usepackage{isomath}
\usepackage{upgreek}
\usepackage{amssymb}
\usepackage{graphicx}

\newcommand{\dotprod}{\, {\scriptzcriptztyle\stackrel{\bullet}{{}}}\,}

\newcommand{\reals}{\mathbf{R}}
\newcommand{\lub}{\mathrm{lub}} 
\newcommand{\glb}{\mathrm{glb}} 
\newcommand{\complex}{\mathbf{C}}
\newcommand{\dom}{\mbox{dom}}
\newcommand{\range}{\mbox{range}}
\newcommand{\cover}{{\mathcal C}}
\newcommand{\integers}{\mathbf{Z}}
\newcommand{\vi}{\, \mathbf{i}}
\newcommand{\vj}{\, \mathbf{j}}
\newcommand{\vk}{\, \mathbf{k}}
\newcommand{\bi}{\, \mathbf{i}}
\newcommand{\bj}{\, \mathbf{j}}
\newcommand{\bk}{\, \mathbf{k}}
\DeclareMathOperator{\Arg}{\mathrm{Arg}}
\DeclareMathOperator{\Ln}{\mathrm{Ln}}
\newcommand{\imag}{\, \mathrm{i}}

\usepackage{graphicx}
\usepackage{color}
%\shadedsolutions
%\definecolor{SolutionColor}{rgb}{1,0.72,0.46} %{0.8,0.9,1}
\newcommand\AM{\textsc{am}}
\newcommand\PM{\textsc{pm}}
     
\newcommand{\quiz}{17}
\newcommand{\term}{Spring}
\newcommand{\due}{Wednesday 27 March  13:20}
\newcommand{\class}{MATH 202, Spring \the\year}
\begin{document}
\large
\vspace{0.1in}
\noindent\makebox[3.0truein][l]{\textbf{\class}}
\textbf{Name:} \hrulefill \\
\noindent \makebox[3.0truein][l]{\textbf{In class work  \quiz}}
\textbf{Row and Seat}:\hrulefill\\
\vspace{0.1in}
\vspace{0.1in}
\noindent  In class work  \textbf{\quiz\/}  has questions \textbf{1} through  \textbf{\numquestions} \/ with a total of \textbf{\numpoints\/}  points.   

\vspace{0.1in}
\noindent  \emph{``The miracle is this: the more we share the more we have.''} 
\hfill \textsc{Leonard Nimoy}


\begin{questions}

\question Use the integral test to decide if each series \emph{diverges} or \emph{converges.}

\begin{parts}

\part [2] $\displaystyle \sum_{k=1}^\infty \frac{\ln(k)}{k^2}$.  You'll need to check that the function $x \mapsto \frac{\ln(x)}{x^2}$ eventually decreases. It doesn't decrease on $[1, \infty)$, but you can show that it eventually decreases.

\begin{solution}%[3.0in]

\end{solution}

\newpage

\part [2] $\displaystyle  \sum_{k=1}^\infty \frac{1}{(k+5)(k+7)}$

\begin{solution}%[4.5in]

\end{solution}

\end{parts}
\end{questions}
\end{document}

\part [2] $\displaystyle \int_1^\infty \frac{1}{x^{\frac{11}{10}}} \, \mathrm{d} x$.  

\begin{solution}[3.0in]

\end{solution}

\part [2] $\displaystyle \int_1^\infty \frac{1}{x^{\frac{9}{10}}} \, \mathrm{d} x$.  

\begin{solution}%[4.5in]

\end{solution}

\end{parts}
\end{questions}
\end{document}

