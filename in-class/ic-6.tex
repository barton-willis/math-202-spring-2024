\documentclass[12pt,fleqn,answers]{exam}
%\usepackage{pifont}
%\usepackage{dingbat,bbding}

\usepackage{amssymb}
\usepackage[intlimits]{amsmath}
\usepackage{epsfig}
\usepackage{upgreek}
\usepackage[super]{nth}
\usepackage[colorlinks=true,linkcolor=black,anchorcolor=black,citecolor=black,filecolor=black,menucolor=black,runcolor=black,urlcolor=black]{hyperref}
\usepackage[letterpaper, margin=0.75in]{geometry}
\addpoints
\boxedpoints
\pointsinmargin
\pointname{pts}
\usepackage{tikz}
\usepackage{tkz-euclide}
\usetikzlibrary{shapes.geometric}
\usetikzlibrary{calc}
\usepackage[final]{microtype}
\frenchspacing
\usepackage[american]{babel}
\usepackage[T1]{fontenc}
\usepackage[]{fourier}
\usepackage{isomath}
\usepackage{upgreek,amsmath}
\usepackage{amssymb}
\usepackage{graphicx}

\newcommand{\dotprod}{\, {\scriptzcriptztyle\stackrel{\bullet}{{}}}\,}

\newcommand{\reals}{\mathbf{R}}
\newcommand{\lub}{\mathrm{lub}} 
\newcommand{\glb}{\mathrm{glb}} 
\newcommand{\complex}{\mathbf{C}}
\newcommand{\dom}{\mbox{dom}}
\newcommand{\range}{\mbox{range}}
\newcommand{\cover}{{\mathcal C}}
\newcommand{\integers}{\mathbf{Z}}
\newcommand{\vi}{\, \mathbf{i}}
\newcommand{\vj}{\, \mathbf{j}}
\newcommand{\vk}{\, \mathbf{k}}
\newcommand{\bi}{\, \mathbf{i}}
\newcommand{\bj}{\, \mathbf{j}}
\newcommand{\bk}{\, \mathbf{k}}
\DeclareMathOperator{\Arg}{\mathrm{Arg}}
\DeclareMathOperator{\Ln}{\mathrm{Ln}}
\newcommand{\imag}{\, \mathrm{i}}

\usepackage{graphicx}
\usepackage{color}
%\shadedsolutions
%\definecolor{SolutionColor}{rgb}{1,0.72,0.46} %{0.8,0.9,1}
\newcommand\AM{\textsc{am}}
\newcommand\PM{\textsc{pm}}
     
\newcommand{\quiz}{6}
\newcommand{\term}{Spring}
\newcommand{\due}{Tuesday 13 Feb 13:20}
\newcommand{\class}{MATH 202, \term \the\year}
\begin{document}
\large
\vspace{0.1in}
\noindent\makebox[3.0truein][l]{\textbf{\class}}
\textbf{Name:} \hrulefill \\
\noindent \makebox[3.0truein][l]{\textbf{In class work week \quiz}}
\textbf{Row and Seat}:\hrulefill\\
\vspace{0.1in}


\noindent  In class work  \textbf{\quiz\/}  has questions \textbf{1} through  \textbf{\numquestions} \/ with a total of \textbf{\numpoints\/}  points.   
Turn in your work at the end of class  \emph{on paper}. This assignment is due \emph{\due}.

\vspace{0.1in}


\begin{questions} 


\question For the DE $\displaystyle  3 y^3   \frac{\mathrm{d} y}{\mathrm{d} x} + 2 x = 0$, do the following:

\begin{parts}

\part [2] Find a GS to the DE.  Remember that a GS has one arbitrary constant.
\begin{solution}[2.5in]
We need to match the given DE to $A^\prime(y)  \frac{\mathrm{d} y}{\mathrm{d} x} = B^\prime(x)$.
To make the match, we need to subtract $2x$ from the DE. That
gives $\displaystyle  3 y^3   \frac{\mathrm{d} y}{\mathrm{d} x} = - 2 x$.
So $A^\prime(y) = 3 y^3$ and $B^\prime(x) = -2 x$.  Integrating, we find that
$A(y) = \frac{3}{4} y^4$ and $B(x) = -x^2 + c$, where $c \in \reals$. So a GS to the DE is
\begin{equation}
    \frac{3}{4}  y^4 = -x^2 + c.
\end{equation}
Another GS is 
\begin{equation}
    3 y^4 = -4 x^2 + \uppi c.
   \end{equation}
And another is
\begin{equation}
    y^4 + 14 = - \frac{4}{3} x^2 + \uppi c.
   \end{equation}
\end{solution}

\part [2] Find a solution to the DE that satisfies $y = 2$ when $x=1$.
\begin{solution}[2.5in]
We need to paste $y=2$ and $x=1$ into a GS. Let's use the GS $   \frac{3}{4}  y^4 = -x^2 + c$.
\begin{equation}
    \frac{3}{4} 2^4 = -1^2 + c.
   \end{equation}
so $c = 13$.  That makes this solution $  \frac{3}{4}  y^4 = -x^2 + 13$.
\end{solution}


\part [2] Use Desmos to graph the solution you found in part `b.' 
\begin{solution}%[2.5in]
\includegraphics*[scale=0.2]{desmos-graph(75).png}
\end{solution}




\end{parts}

\newpage 
\question Find the numerical value of the definite integral 
$\int_4^8 \frac{1}{5 x + 1} \, \mathrm{d} x$.

\begin{solution}
Let $z = 5 x + 1$. Then $\mathrm{d} z = 5 \mathrm{d} x$. When $x=4$, 
we have $z = 21$; and when $x=8$, we have $z = 41$. So
\begin{equation}
    \int_4^8 \frac{1}{5 x + 1} \, \mathrm{d} x
    = \int_{21}^{41} \frac{1}{5} \frac{1}{z} \, \mathrm{d} z
    = \frac{1}{5} \left( \ln(41) - \ln(21) \right).
\end{equation}
Most folks would say that using the logarithm of a sum rule, a 
simpler answer is
\begin{equation}
    \int_4^8 \frac{1}{5 x + 1} \, \mathrm{d} x
     = \frac{1}{5}  \ln\left( \frac{41}{21} \right).
\end{equation}
We could use the identity $y \ln(x) = \ln(x^y)$. That gives an answer
\begin{equation}
   \ln\left( \sqrt[5]{\frac{41}{21}} \right),
\end{equation}
but trading a division by $5$ by a fifth root doesn't seem like a 
simplification.
\end{solution}

\end{questions}

\end{document}

