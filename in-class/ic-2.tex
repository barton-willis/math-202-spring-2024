\documentclass[12pt,fleqn,answers]{exam}
%\usepackage{pifont}
%\usepackage{dingbat,bbding}

\usepackage{amssymb}
\usepackage[intlimits]{amsmath}
\usepackage{epsfig}
\usepackage{upgreek}
\usepackage[super]{nth}
\usepackage[colorlinks=true,linkcolor=black,anchorcolor=black,citecolor=black,filecolor=black,menucolor=black,runcolor=black,urlcolor=black]{hyperref}
\usepackage[letterpaper, margin=0.75in]{geometry}
\addpoints
\boxedpoints
\pointsinmargin
\pointname{pts}
\usepackage{tikz}
\usepackage{tkz-euclide}
\usetikzlibrary{shapes.geometric}
\usetikzlibrary{calc}
\usepackage[final]{microtype}
\frenchspacing
\usepackage[american]{babel}
\usepackage[T1]{fontenc}
\usepackage[]{fourier}
\usepackage{isomath}
\usepackage{upgreek,amsmath}
\usepackage{amssymb}
\usepackage{graphicx}

\newcommand{\dotprod}{\, {\scriptzcriptztyle\stackrel{\bullet}{{}}}\,}

\newcommand{\reals}{\mathbf{R}}
\newcommand{\lub}{\mathrm{lub}} 
\newcommand{\glb}{\mathrm{glb}} 
\newcommand{\complex}{\mathbf{C}}
\newcommand{\dom}{\mbox{dom}}
\newcommand{\range}{\mbox{range}}
\newcommand{\cover}{{\mathcal C}}
\newcommand{\integers}{\mathbf{Z}}
\newcommand{\vi}{\, \mathbf{i}}
\newcommand{\vj}{\, \mathbf{j}}
\newcommand{\vk}{\, \mathbf{k}}
\newcommand{\bi}{\, \mathbf{i}}
\newcommand{\bj}{\, \mathbf{j}}
\newcommand{\bk}{\, \mathbf{k}}
\DeclareMathOperator{\Arg}{\mathrm{Arg}}
\DeclareMathOperator{\Ln}{\mathrm{Ln}}
\newcommand{\imag}{\, \mathrm{i}}

\usepackage{graphicx}
\usepackage{color}
%\shadedsolutions
%\definecolor{SolutionColor}{rgb}{1,0.72,0.46} %{0.8,0.9,1}
\newcommand\AM{\textsc{am}}
\newcommand\PM{\textsc{pm}}
     
\newcommand{\quiz}{2}
\newcommand{\term}{Spring }
\newcommand{\due}{Tuesday 30 January 13:20}
\newcommand{\class}{MATH 202, \term \the\year}
\begin{document}
\large
\vspace{0.1in}
\noindent\makebox[3.0truein][l]{\textbf{\class}}
\textbf{Name:} \hrulefill \\
\noindent \makebox[3.0truein][l]{\textbf{In class work \quiz}}
\textbf{Row and Seat}:\hrulefill\\
\vspace{0.1in}


\noindent  In class work  \textbf{\quiz\/}  has questions \textbf{1} through  \textbf{\numquestions} \/ with a total of \textbf{\numpoints\/}  points.   
Turn in your work at the end of class  \emph{on paper}. This assignment is due \emph{\due}.

\vspace{0.1in}


\begin{questions} 

\question[5] Let $b$ be a positive number. Find the length of the curve $y = b \left(\frac{x}{b}\right)^{3/2}$ where $0 \leq x \leq b$.  Possibly you will be more successful if you begin by expressing $y$ in an equivalent form:  $y = b^{-1/2} x^{3/2}$
\textbf{Hint:} The quantity $b$ is a length. To make the dimensions correct, the answer \emph{must} be of the form $ b \times \mbox{number}$; if not, check your work!


\begin{solution}[2.5in]  A better alternative to $y = a \left(\frac{x}{a}\right)^{3/2}$
    might be $y = \frac{x^{3/2}} {\sqrt{a}}$. We have
    \begin{equation}
      \frac{\mathrm{d} y}{\mathrm{d} x} = \frac{3}{2 \sqrt{a}} x^{1/2}.
    \end{equation}
    That makes
    \begin{equation}
        1+ \left(\frac{\mathrm{d} y}{\mathrm{d} x} \right)^2 = 
        1 + \frac{9 x}{4 a}
      \end{equation}
    The arclength $s$ is
    \begin{align*}
        s &= \int_0^a \sqrt{1 + \frac{9 x}{4 a}} \,\mathrm{d} x,\\
        \intertext{Let's substitute $z = 1 + \frac{9 x}{4 a}$. Then
        $\mathrm{d} z = \frac{9}{4 a} \mathrm{d} x$. Solving
        for $\mathrm{d} x$ gives $\mathrm{d} x = \frac{4 a}{9} \mathrm{d} z$.
        And one more detail: we know the limits of integration  for $x$, but
        we need them for $z$. When $x=0$, we have $z = 1$. And when 
        $x=a$, we have $z = 1 + \frac{9}{4} =\frac{13}{4}$. We're ready:
        }  \\
           &= \frac{4 a}{9} \int_1^{13/4} \sqrt{z} \,\mathrm{d} z, \\
           &= \left. \frac{3}{2} z^{3/2} \right \vert_{z= 1}^{z = 13/4}, \\
           &= \frac{4 a}{9} \frac{3}{2} \left(  \left(\frac{13}{4} \right)^{3/2} -1 \right), \\
    \end{align*}
 

\end{solution}    

\newpage
\question[5] Find the surface area of
the solid generated by rotating the curve $y = \sqrt{x}$
 where $0 \leq x \leq 1$ about the x-axis.  

\begin{solution}
A better alternative to $y = a \sqrt{\frac{x}{a}}$
might be $y = \sqrt{a} \sqrt{x}$. We have
\begin{equation}
  \frac{\mathrm{d} y}{\mathrm{d} x} = \frac{1}{2} \sqrt{a} x^{-1/2}.
\end{equation}
So 
\begin{equation}
  \sqrt{1 + \left(\frac{\mathrm{d} y}{\mathrm{d} x}\right)^2}
     = \sqrt{1 + \frac{a}{4 x}}.
\end{equation}
The surface area is 

\begin{align*}
  \text{Area} &= 2 \uppi \int_0^a  \sqrt{a} \sqrt{x} 
  \sqrt{1 + \frac{1}{4 x}} \, \mathrm{d} x \\
  &= 2 \uppi \sqrt{a} \int_0^a \sqrt{x + \frac{a}{4}} \, \mathrm{d} x \\
  &= \frac{\uppi}{6} \left(5^{3/2} - 1 \right) a^2.
\end{align*}
\end{solution}
  \end{questions}


\end{document}

