\documentclass[12pt,fleqn]{exam}
%\usepackage{pifont}
%\usepackage{dingbat,bbding}

\usepackage{amssymb}
\usepackage[intlimits]{amsmath}
\usepackage{epsfig}
\usepackage{upgreek}
\usepackage[super]{nth}
\usepackage[colorlinks=true,linkcolor=black,anchorcolor=black,citecolor=black,filecolor=black,menucolor=black,runcolor=black,urlcolor=black]{hyperref}
\usepackage[letterpaper, margin=0.75in]{geometry}
\addpoints
\boxedpoints
\pointsinmargin
\pointname{pts}
\usepackage{tikz}
\usepackage{tkz-euclide}
\usetikzlibrary{shapes.geometric}
\usetikzlibrary{calc}
\usepackage[final]{microtype}
\frenchspacing
\usepackage[american]{babel}
\usepackage[T1]{fontenc}
\usepackage[]{fourier}
\usepackage{isomath}
\usepackage{upgreek,amsmath}
\usepackage{amssymb}
\usepackage{graphicx}

\newcommand{\dotprod}{\, {\scriptzcriptztyle\stackrel{\bullet}{{}}}\,}

\newcommand{\reals}{\mathbf{R}}
\newcommand{\lub}{\mathrm{lub}} 
\newcommand{\glb}{\mathrm{glb}} 
\newcommand{\complex}{\mathbf{C}}
\newcommand{\dom}{\mbox{dom}}
\newcommand{\range}{\mbox{range}}
\newcommand{\cover}{{\mathcal C}}
\newcommand{\integers}{\mathbf{Z}}
\newcommand{\vi}{\, \mathbf{i}}
\newcommand{\vj}{\, \mathbf{j}}
\newcommand{\vk}{\, \mathbf{k}}
\newcommand{\bi}{\, \mathbf{i}}
\newcommand{\bj}{\, \mathbf{j}}
\newcommand{\bk}{\, \mathbf{k}}
\DeclareMathOperator{\Arg}{\mathrm{Arg}}
\DeclareMathOperator{\Ln}{\mathrm{Ln}}
\newcommand{\imag}{\, \mathrm{i}}

\usepackage{graphicx}
\usepackage{color}
%\shadedsolutions
%\definecolor{SolutionColor}{rgb}{1,0.72,0.46} %{0.8,0.9,1}
\newcommand\AM{\textsc{am}}
\newcommand\PM{\textsc{pm}}
     
\newcommand{\quiz}{11}
\newcommand{\term}{Spring}
\newcommand{\due}{Thursday February 29 13:20}
\newcommand{\class}{MATH 202, Fall \the\year}
\begin{document}
\vspace{0.1in}
\noindent\makebox[3.0truein][l]{\textbf{\class}}
\textbf{Name:} \hrulefill \\
\noindent \makebox[3.0truein][l]{\textbf{In class work  \quiz}}
\textbf{Row and Seat}:\hrulefill\\
\vspace{0.1in}
\small
\vspace{0.1in}

\noindent \emph{``There is nothing more precious than laughter–it is strength to laugh and lose oneself, to be light.''} \hfill \textsc{Frida Kahlo}

\vspace{0.1in}

\noindent  In class work  \textbf{\quiz\/}  has questions \textbf{1} through  \textbf{\numquestions} \/ with a total of \textbf{\numpoints\/}  points.   
Turn in your work at the end of class  \emph{on paper}. This assignment is due \emph{\due}.

\vspace{0.1in}

Here are some results that you might like to use
\begin{align*}
% {{\cos{(x)}}^{2}} &=\frac{\cos{\left( 2 x\right) }}{2}+\frac{1}{2},\\
 %{{\cos{(x)}}^{4}} &=\frac{\cos{\left( 4 x\right) }}{8}+\frac{\cos{\left( 2 x\right) }}{2}+\frac{3}{8}, \\
% {{\sin{(x)}}^{2}} &=\frac{1}{2}-\frac{\cos{\left( 2 x\right) }}{2}, \\
% {{\cos{(x)}}^{2}}\, {{\sin{(x)}}^{2}}&=\frac{1}{8}-\frac{\cos{\left( 4 x\right) }}{8},\\
% {{\cos{(x)}}^{4}}\, {{\sin{(x)}}^{2}} &=-\frac{\cos{\left( 6 x\right) }}{32}-\frac{\cos{\left( 4 x\right) }}{16}+%\frac{\cos{\left( 2 x\right) }}{32}+\frac{1}{16},\\ 
% {{\sin{(x)}}^{4}}&=\frac{\cos{\left( 4 x\right) }}{8}-\frac{\cos{\left( 2 x\right) }}{2}+\frac{3}{8}, \\ 
 {{\cos{(x)}}^{2}}\, {{\sin{(x)}}^{4}} &=\frac{\cos{\left( 6 x\right) }}{32}-\frac{\cos{\left( 4 x\right) }}{16}-\frac{\cos{\left( 2 x\right) }}{32}+\frac{1}{16}, \\
 {{\cos{(x)}}^{4}}\, {{\sin{(x)}}^{4}} &=\frac{\cos{\left( 8 x\right) }}{128}-\frac{\cos{\left( 4 x\right) }}{32}+\frac{3}{128}.
\end{align*}
\begin{questions}

    \question[2]  Use Desmos to sketch the region $Q$ defined as $Q = \{(x,y) \mid 0 \leq y \leq x^4 \sqrt{1-x^2}
    \mbox{ and } 0 \leq x \leq 1 \}$. Duplicate
    the graph here.
    \begin{solution}[2.0in]
   % \includegrap
    \end{solution}
    
    \question[2] Find $\mbox{area}(Q)$. \textbf{Suggestion:} Substitute $x = \sin(\theta)$. When you change variables,
    also change  the limits of integration; for example, when $x=1$, we have $\theta = \frac{\uppi}{2}$.
  \begin{solution}%[2.0in]
    \end{solution}
        \newpage
        \question[2] Using your graph, make a pretty good guess for the x-coordinate to the centroid of $Q$.
  \begin{solution}[1.0in]
    \end{solution}
    

    
       \question[2] Find the x-coordinate of the centroid of $Q$.
  \begin{solution}[2.0in]
    \end{solution}
\end{questions}

\end{document}

