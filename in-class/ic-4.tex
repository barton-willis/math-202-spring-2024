\documentclass[12pt,fleqn]{exam}
%\usepackage{pifont}
%\usepackage{dingbat,bbding}

\usepackage{amssymb}
\usepackage[intlimits]{amsmath}
\usepackage{epsfig}
\usepackage{upgreek}
\usepackage[super]{nth}
\usepackage[colorlinks=true,linkcolor=black,anchorcolor=black,citecolor=black,filecolor=black,menucolor=black,runcolor=black,urlcolor=black]{hyperref}
\usepackage[letterpaper, margin=0.75in]{geometry}
\addpoints
\boxedpoints
\pointsinmargin
\pointname{pts}
\usepackage{tikz}
\usepackage{tkz-euclide}
\usetikzlibrary{shapes.geometric}
\usetikzlibrary{calc}
\usepackage[final]{microtype}
\frenchspacing
\usepackage[american]{babel}
\usepackage[T1]{fontenc}
\usepackage[]{fourier}
\usepackage{isomath}
\usepackage{upgreek,amsmath}
\usepackage{amssymb}
\usepackage{graphicx}

\newcommand{\dotprod}{\, {\scriptzcriptztyle\stackrel{\bullet}{{}}}\,}

\newcommand{\reals}{\mathbf{R}}
\newcommand{\lub}{\mathrm{lub}} 
\newcommand{\glb}{\mathrm{glb}} 
\newcommand{\complex}{\mathbf{C}}
\newcommand{\dom}{\mbox{dom}}
\newcommand{\range}{\mbox{range}}
\newcommand{\cover}{{\mathcal C}}
\newcommand{\integers}{\mathbf{Z}}
\newcommand{\vi}{\, \mathbf{i}}
\newcommand{\vj}{\, \mathbf{j}}
\newcommand{\vk}{\, \mathbf{k}}
\newcommand{\bi}{\, \mathbf{i}}
\newcommand{\bj}{\, \mathbf{j}}
\newcommand{\bk}{\, \mathbf{k}}
\DeclareMathOperator{\Arg}{\mathrm{Arg}}
\DeclareMathOperator{\Ln}{\mathrm{Ln}}
\newcommand{\imag}{\, \mathrm{i}}

\usepackage{graphicx}
\usepackage{xcolor}
\shadedsolutions
%\definecolor{shadecolor}{RGB}{200,200,200}
\colorlet{shadecolor}{gray!10}
%\colorlet{shadecolor}{gray!40}
\newcommand\AM{\textsc{am}}
\newcommand\PM{\textsc{pm}}
     
\newcommand{\quiz}{4}
\newcommand{\term}{Spring }
\newcommand{\due}{Tuesday 6 February 13:20}
\newcommand{\class}{MATH 202, \term \the\year}
\begin{document}
\large
\vspace{0.1in}
\noindent\makebox[3.0truein][l]{\textbf{\class}}
\textbf{Name:} \hrulefill \\
\noindent \makebox[3.0truein][l]{\textbf{In class work \quiz}}
\textbf{Row and Seat}:\hrulefill\\
\vspace{0.1in}


\noindent  In class work  \textbf{\quiz\/}  has questions \textbf{1} through  \textbf{\numquestions} \/ with a total of \textbf{\numpoints\/}  points.   
Turn in your work at the end of class  \emph{on paper}. This assignment is due \emph{\due}.

\vspace{0.1in}


\begin{questions} 

 \question [1] My friend Morwenna claims that $\int_{-9}^9 x \sqrt{1+x^2} \, \mathrm{d} x = 0$, but she doesn't 
 know why this is true.  Explain to Morwenna why it is true that $\int_{-9}^9 x \sqrt{1+x^2} \, \mathrm{d} x = 0$.
 
 \begin{solution}[1.5in]
 
 \end{solution}
 
 \question [1] My friend Louisa claims that because the interval $-9$ to $9$ is symmetric with respect to the 
 origin,  that $\int_{-9}^9 (x^2-x) \, \mathrm{d} x  = 0$.   Explain to Louisa what condition she is missing.
  \begin{solution}[1.5in]
 
 \end{solution}
 \question [1] My friend Mr. Bert Frogmore is having difficulty evaluating the integral $\int_{-1}^1 \sqrt{1-x^2} \, \mathrm{d} x$.  Show  Mr. Frogmore an easy (and I mean easy) way of finding the numerical value of this definite integral.
  \begin{solution}%[1.5in]
 
 \end{solution}
 \newpage
 
\question For the region of the xy plane $Q = \{(x,y) | 0 \leq y \leq 2 -x, \mbox{ and } 0 \leq x \leq 2\}$, do the following

\begin{parts}

\part Sketch the region $Q$. 
 \begin{solution}[1.5in]
 
 \end{solution}
 
 \part[1] Find the \emph{area} of $Q$. 
 \begin{solution}[2.05in]
 
 \end{solution}
 
 \part[1] Find the x-coordinate of the centroid  of $Q$. 
 \begin{solution}[2.5in]
 
 \end{solution}
 
  \part[1] Find the y-coordinate of the centroid  of $Q$. 
 \begin{solution}%[2.5in]
 
 \end{solution}
\end{parts}  
\end{questions}

\end{document}

